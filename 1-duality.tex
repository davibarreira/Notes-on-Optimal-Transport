\newpage
\section{Duality of the Kantorovich Problem}

In this section we deal with Duality Theorems regarding the Kantorovich Problem.
Under some conditions, the original Kantorovich Problem (Primal) is equivalent to a Dual
formulation, where instead of minimizing transport plans, one seeks to maximize potentials.
Hence, we'll begin this section by introducing the notion of the Dual Problem, and then we'll prove the
equivalence between the Dual and the Primal, starting from more restrictive conditions (e.g. compact spaces)
and moving to more general conditions (e.g. Polish spaces). We finish the section with the celebrated
Kantorovich-Rubinstein's Duality Theorem.

Before introducing the Dual Problem, we need the following result:
\begin{lemma}

  The Kantorovich Problem (\ref{def:KP}) is equivalent to:
\begin{align}
  \inf_{\gamma \in \mathcal M_+(X\times Y)}
  \int_{X \times Y} c(x,y)d\gamma &+
  \sup_{(\phi,\psi) \in B}
  \int_X \phi(x) \ d\mu
  \nonumber
  \\
  &+ \int_Y \psi(y) \ d\nu
  - \int_{X\times Y} \phi(x) + \psi(y) \ d\gamma
  \label{eq:KP2}
\end{align}
Where $B := \{\phi \in C_b(X) \ \mathrm{and} \ \psi \in C_b(Y)\}$.
\end{lemma}
\begin{prf}
  Let's suppose that $\gamma \notin \Pi(\mu,\nu)$.
  Then, without lost of generality, $\exists A \ : \ \mu(A) \neq
    \gamma(A,Y)$. Hence, can make $\phi(x) = M$ in $A$ and null elsewhere.
  So,
  \begin{equation*}
    \int_A \phi \ d\mu - \int_A \phi \ d\gamma	= M(\mu(A)-\gamma(A,Y))
  \end{equation*}
  Since we can make $M$ arbitrarily large or small, we conclude that
  \begin{equation*}
    \sup_{(\phi,\psi) \in B}
    \int_X \phi(x) \ d\mu + \int_Y \psi(y) \ d\nu -
    \int_{X\times Y} \phi(x) + \psi(y) \ d\gamma = +\infty
  \end{equation*}

  This implies that for $\gamma \notin \Pi(\mu,\nu)$, equation \eqref{eq:KP2}
  is $+\infty$. If $\gamma \in \Pi(\mu,\nu)$, then we return to
  \begin{equation*}
    \inf_{\gamma \in \Pi(\mu,\nu)} \int_{X\times Y} c \ d\gamma
  \end{equation*}
  With this, we proved that the argument that minimizes
  equation \eqref{eq:KP2} must
  be inside $\{\gamma \in \Pi(\mu,\nu)\}$, which is the original Kantorovich
  Problem.
\end{prf}

\vspace{5mm}

With (KP) reformulated, the Dual Problem consists of exchanging the order
of the $\inf$ and the $\sup$:
\begin{itemize}
  \item \textbf{Primal}
        \footnote{$(\phi\oplus \psi) (x,y) = \phi(x) + \psi(y)$}:
        \begin{equation}
          \inf_{\gamma \in \mathcal M_+(X\times Y)}
          \sup_{(\phi,\psi) \in B}
          \int_{X \times Y} c \ d\gamma +
          \int_X \phi \ d\mu + \int_Y \psi \ d\nu -
          \int_{X\times Y} \phi \oplus \psi \ d\gamma
        \end{equation}

  \item \textbf{Dual}:
        \begin{equation}
          \sup_{(\phi,\psi) \in B}
          \inf_{\gamma \in \mathcal M_+(X\times Y)}
          \int_{X \times Y} c \ d\gamma +
          \int_X \phi \ d\mu + \int_Y \psi \ d\nu -
          \int_{X\times Y} \phi \oplus \psi \ d\gamma
        \end{equation}
\end{itemize}

Note that in the Dual formulation, we can rewrite it as:
\begin{equation}
  \sup_{(\phi,\psi)\in B}
  \int_X \phi \ d\mu + \int_Y \psi \ d\nu -
  \inf_{\gamma \in \mathcal M_+(X\times Y)}
  \int_{X\times Y} c - (\phi \oplus \psi) \ d\gamma
\end{equation}

If there exists an $A$ such that for all $\forall (x,y) \in A, \ \phi(x) + \psi(y) \geq c(x,y)$, then
$\inf_\gamma \int c - (\phi \oplus \psi) \ d\gamma = -\infty$
since we can choose any $\gamma \in \mathcal M_+(X\times Y)$.

Therefore, we can formally state the Dual Problem as:
\begin{definition}
  Given $\mu \in \mathcal P(X)$, $\nu \in \mathcal P (Y)$ and
  a cost $c:X \times Y \to \mathbb R_+$. The
  Dual Problem is given by
\end{definition}
\begin{flalign}
  \mathrm{(DP)} &&
  \sup \left \{
  \int_X \phi \ d\mu + \int_Y \psi \ d\nu \ :
  \phi \in C_b(X) \ , \psi \in C_b(y) \ ,
  \ \phi \oplus \psi \leq c
  \right \}
  &&
  \label{eqt:dualproblem}
\end{flalign}

We call \textbf{Weak Duality} if
$\mathrm{(DP)} \leq \mathrm{(KP)}$, and we call \textbf{Strong Duality}
if
$\mathrm{(DP)} = \mathrm{(KP)}$.
One can easily prove that for (KP), the Weak Duality is always true.
The more interesting question is ``When does one have Strong Duality?''.

\begin{lemma}
  The Dual Problem for the Kantorovich Problem always satisfies the
  Weak Duality, i.e.
$\mathrm{(DP)} \leq \mathrm{(KP)}$.
\end{lemma}

\begin{prf}
  Since $\phi \oplus \psi \leq c$. Therefore,
  \begin{equation*}
    \int_X \phi \ d\mu +
    \int_Y \psi \ d\nu
    =
    \int_{X \times Y} \phi \oplus \psi \ d\gamma \leq
    \int_{X\times Y} c(x,y) \ d\gamma
  \end{equation*}
\end{prf}

Before starting the proof of duality, we must introduce the concepts
of $c$-transform and $c$-Cyclical monotonicity.

\begin{definition}(c-Transform)
  Given $f: X \to \overline{\mathbb R}$, and
  $c:X\times Y \to \overline{\mathbb R}$,
  the $c$-transform of $f$ is:
  \begin{equation}
    f^c(y) := \inf_x c(x,y) - f(x)
  \end{equation}
  Function $f^c$ is also called the $c$-conjugate of $f$. Moreover,
  we say that $f$ is $c$-concave if
  $\exists \ g:Y\to \overline{\mathbb R}$
  such that $g^c(x) = f(x)$.
  \label{def:c-transform}

  Note that the $c$-transform is a generalization of the
  Legendre-Fenchel transform, which is defined as:
  \begin{equation}
    f^*(y) := \sup_x x \cdot y - f(x)
  \end{equation}
\end{definition}

\begin{lemma}
  Let $c: X \times Y \to \overline{\mathbb R}$ be uniformly continuous. Define two functions
  $\phi:X \to \mathbb R$ and $\psi : Y \to \mathbb R$
  Therefore, $\phi^c$ and $\psi^c$ have the same modulus of continuity
  \footnote{Check Theorem \ref{thm:mod_continuity} for the definition of modulus of continuity}
  as $c$.
  \label{lem:cunif}
\end{lemma}
\begin{prf}
  By Theorem \ref{thm:mod_continuity}, there exists a modulus
  of continuity $\omega$, such that
  \begin{equation*}
    |c(x,y) - c(x',y')| \leq \omega(d(x,x')+d(y,y'))
  \end{equation*}
  Observe that for $g_x(y) = c(x,y) - \phi(x)$
  \begin{equation*}
    |g_x(y) - g_x(y')|=|c(x,y) - c(x,y')| \leq
    \omega(d(x,x)+d(y,y')) = \omega(d(y,y'))
  \end{equation*}
  Hence, $g_x$ has modulus of continuity $\omega$. Now, using the
  Inf-Sup Inequality \ref{lem:infsup_ineq}
  \begin{align*}
    |\inf_x g_x(y) - \inf_x g_x(y')| & =
    |\phi^c(y) - \phi^c(y')| \leq \sup_x |g_x(y) - g_x(y')| =                          \\
                                     & =\sup_x |c(x,y) - c(x,y')| \leq \omega(d(y,y'))
  \end{align*}

  Using the same argument for $\psi^c$, we showed that both
  $c-$transforms have the same modulus of continuity.

\end{prf}

With the definition of $c$-transforms and the lemma above, we can prove the following theorem:

\begin{theorem}(Santambrogio 1.11)

  For $X$ and $Y$ compact metric spaces, and $c:X \times Y \to
    \overline{\mathbb R}$ continuous. Then, the Dual Problem
  has a solution $(\phi,\phi^c)$ for $\phi$ $c-$concave. Hence
  \begin{equation}
    \max(\mathrm{DP}) =
    \max_{\phi \in c-conc.(X)} \int_X \phi \ d\mu +
    \int_Y \phi^c \ d\nu
  \end{equation}
  \label{thm:c-conc}
\end{theorem}
\begin{prf}
  Let $(\phi_n,\psi_n)$ be a maximizing sequence of the Dual problem.
  Note that the $c$-transforms always improve the Dual Problem, since
  $\phi_n\oplus \psi_n \leq c$, which implies that
  \begin{align*}
    \phi_n^c(y):= \inf_x c(x,y) - \phi_n(x) \geq \psi_n(y) \\
    \psi_n^c(x):= \inf_y c(x,y) - \psi_n(y) \geq \phi_n(x) \\
    \int_X \phi_n \ d\mu +
    \int_Y \psi_n \ d\nu \leq
    \int_X \phi_n \ d\mu +
    \int_Y \phi_n^c \ d\nu
  \end{align*}
  Hence, the sequence $(\phi_n, \phi^c_n)$ is also maximizing.

  Since $X \times Y$ is compact, the cost $c$ is uniformly continuous. Therefore,
  by Lemma \ref{lem:cunif}, the $c-$transforms of $\phi_n$ and $\psi_n$ are bounded by the
  same modulus of continuity $\omega$ as the cost function $c$.

  Instead of using
  \begin{equation*}
    \psi^c_n (x) = \inf_y c(x,y) - \psi(y)
  \end{equation*}
  We will use
  \begin{equation*}
    \psi^c_n (x) := \inf_y c(x,y) - \phi_n^c(y) = \phi_n^{c c}(x)
  \end{equation*}
  This sequence is still maximizing, since
  \begin{align*}
    \phi_n^c(y)  = \inf_x c(x,y) - \phi_n(x) \geq \psi_n(y) &\implies 
    \phi_n(x) + \phi_n^c(y) \leq c(x,y)                        \\
    &\implies
    \psi_n^c(x)  = \inf_y c(x,y) - \phi_n^c(y) \geq \phi_n(x)
  \end{align*}

  Therefore, for a maximizing sequence $(\phi_n,\psi_n)$, we can
  instead take the maximizing sequence
  $(\psi^c_n,\phi^c_n)=(\phi^{c c}_n,\phi^c_n)$.

  Our goal now is to use the Àrzela-Ascoli Theorem (\ref{thm:arzela-ascoli}), so
  we can take a subsequence converging uniformly. To use the theorem, we'll
  show that our sequence
  $(\psi^c_n,\phi^c_n)$ is Equicontinuous (see Definition \ref{def:equicontinuous})
  and Equibounded (see definition \ref{def:equibounded}).

  First, note that $(\psi^{c}_n,\phi^c_n)$
  is in fact Equicontinuous, since
  for any $\epsilon > 0$, we can take $\delta >0$ such that
  $d(y,y') < \delta \implies w(d(y,y')) < \epsilon$ and
  $|\phi_n^c(y) - \phi_n^c(y')| \leq w(d(y,y')) < \epsilon$, for every
  $n \in \mathbb N$.
  
  Next, let's prove that the sequence is Equibounded. Taking the supremum of the inequality, we obtain
  \begin{equation*}
    \sup_{y,y'} |\phi^c_n(y) - \phi^c_n(y')| \leq
    \sup_{y,y'}w(d(y,y')) = w(\text{diam}(Y))
  \end{equation*}
  The equality in the equation above is true because the function
  $\omega$ is increasing, and the set $Y$ is compact. Again, the
  same argument works for $\psi_n^c$.

  Next, realize that we can add and subtract constants from
  the Dual Problem without modifying the results:

  \begin{equation*}
    \int_X \psi_n^c \ d\mu + \int_Y \phi_n^c \ d\nu =
    \int_X \psi_n^c + C_n \ d\mu + \int_Y \phi_n^c - C_n \ d\nu
  \end{equation*}

  Let's take $C_n = \min_y \phi_n^c(y)$. We now change the sequence
  of functions to $(\psi_n^c + C_n, \phi_n^c - C_n)$, which preserves
  the maximizing property. Note that $\min_y \phi_n^c - C_n = 0$.
  Hence,

  \begin{align*}
    \sup_{y,y'} |\phi^c_n(y) - \phi^c_n(y')| =
    \max_y \phi_n^c(y) - \min_y \phi_n^c(y) =
    \max_y \phi_n^c(y) \leq \omega(\text{diam}(Y))
  \end{align*}
  Also, for any $x \in X$:
  \begin{align*}
    \psi_n^c(x) = \inf_y c(x,y) - \phi^c_n(y) \in
    [\min_y  \ c(x,y) - \omega(\text{diam}(Y)),\max_y  \ c(x,y) \ ]
  \end{align*}

  With this, we showed that the sequence is Equibounded. Therefore,
  since we are on a compact set and the sequence
  $(\psi_n^c, \phi_n^c)$ is both Equicontinuous and Equibounded,
  we can apply the Àrzela-Ascoli Theorem \ref{thm:arzela-ascoli}.
  Thus, we can obtain a subsequence
  $(\psi_{n_k}^c,\phi_{n_k}^c)$ that converges uniformly to
  $(\psi,\phi)$. As a consequence of this uniform convergence

  \begin{equation*}
    \int_X \psi_{n_k}^c \ d\mu +
    \int_Y \phi_{n_k}^c \ d\nu
    \to
    \int_X \phi \ d\mu +
    \int_Y \psi \ d\nu
  \end{equation*}
  With this, we proved that there exists a pair of functions
  $(\phi, \psi)$ that are the limits of a maximizing sequence and that
  satisfy the constraint (i.e. $\phi(x)+\psi(y) \leq c(x,y)$), hence,
  the Dual Problems has a solution. Also, since $\phi^c \geq \psi$,
  then $(\phi,\phi^c)$ is also an optimal solution for the Dual, and this maximization
  problem can be restricted to searching in $c$-concave functions, i.e.:
  \begin{equation*}
    \max(\mathrm{DP}) =
    \max_{\phi \in c-conc.(X)} \int_X \phi \ d\mu +
    \int_Y \phi^c \ d\nu
  \end{equation*}
\end{prf}

\vspace{5mm}

When Strong Duality is true, the functions $\phi, \psi$ that maximize the Dual Problem
are called the \textbf{Kantorovich Potentials}.
We haven't yet proved that $\mathrm{\max(DP)}=\mathrm{\min(KP)}$, the theorem above
only gave us an idea of how the solution of the Dual Problem looks-like. Before proving
our first theorem on Strong Duality, we'll need a bit more definitions
and results.

\begin{definition}(Cyclic Monotonicity)
  For $c:X \times Y \to \overline{\mathbb R}$, a set $\Gamma \subset
    X \times Y$ is called $c$-cyclical monotone (c-CM) if
  $\forall n \in \mathbb N$ and $(x_i,y_i) \in \Gamma$ for
  $i \in \{1,...,n\}$
  \begin{equation}
    \sum^n_{i=1}c(x_i,y_i) \leq
    \sum^n_{i=1} c(x_i,y_{\sigma(i)})
  \end{equation}
  Where $\sigma(i)$ is a permutation of the indexes.
  \label{def:cyclic-monotonicity}
\end{definition}
Note that
this is a stronger property than monotonicity, since for
$n=2$ and $c(x,y) = \langle x, y \rangle$, if $\Gamma$ is c-CM,
then monotonicity is satisfied:
\begin{equation}
  \langle x_1,y_1 \rangle + \langle x_2, y_2 \rangle \leq
  \langle x_1, y_2, \rangle + \langle x_2, y_1 \rangle
\end{equation}

\begin{definition}
  For $X$	a separable metric space, we define the support of a
  a measure $\mu$ as
  \begin{equation}
    \text{spt } \mu := \bigcap
    \{
    A \ : \ A \text{ is closed and } \mu(X\setminus A) =0
    \}
  \end{equation}
\end{definition}

We can now give an overview of the proof
of first Strong Duality Theorem. The proof consists of showing
that for an optimal
plan $\gamma$, its support $\text{spt} (\gamma)$ is $c$-CM and
that for a $c$-CM set there exists a $c$-concave function
$\phi(x)$ such that $\phi(x)+\phi^c(y) = c(x,y)$ for
$(x,y) \in \text{spt}(\gamma)$. Hence, this would prove that
\begin{equation}
  \int_{X\times Y} c(x,y) \ d\gamma = \int_X \phi(x)\ d\mu + \int_Y \phi^c(y)d\nu
\end{equation}

\begin{theorem}(Santambrogio 1.37)
  If $\Gamma \neq \varnothing $ and is $c$-CM with
  $c:X\times Y \to \mathbb R$. Then,
  there exists a $c$-concave function
  $\phi:X \to \mathbb R \cup \{-\infty\}$ (different than the constant value $-\infty$) such that
  \begin{equation}
    \Gamma \subset \{
    (x,y) \ : \ \phi(x)+\phi^c(y) = c(x,y)
    \}
  \end{equation}
  In other words,
  $\forall x,y \in \Gamma, \ c(x,y) = \phi(x) + \phi^c(y)$.
  \label{thm:existsPhic}
\end{theorem}

\begin{prf}
  Fix a point $(x_0,y_0) \in \Gamma$. For $x \in X$, let
  \begin{align*}
    \phi(x) & := \inf\{
    c(x,y_n) - c(x_n,y_n) + c(x_n,y_{n-1})-c(x_{n-1},y_{n-1})+...+
    \\
            & +c(x_1,y_0)-c(x_0,y_0) \ : \ n \in \mathbb N,
    (x_i,y_i) \in \Gamma \ \forall i=1,...,n
    \}
    \\
    \\
    \psi(y) & :=
    -\inf \{
    -c(x_n,y)+c(x_n,y_{n-1}) - c(x_{n-1},y_{n-1})+...+
    \\
            & c(x_1,y_0)
    -c(x_0,y_0)) \ : \ n\in \mathbb N, (x_i,y_i) \in \Gamma \
    \forall i = 1,...,n, y_n = y
    \}
  \end{align*}
  Note that if $y \notin (\pi_y)(\Gamma)$, then there is no
  $(x_n,y) = (x_n,y_n) \in \Gamma$. Therefore,
  \begin{equation*}
    \psi(y) = -\inf\{\varnothing\} = -\infty
  \end{equation*}
  This implies that $\psi(y)> -\infty \iff y \in (\pi_y)(\Gamma)$. Note that:
  \begin{align*}
    \psi^c(x) & = \inf_y c(x,y) - \psi(y) =
    \inf_{y \in (\pi_y)(\Gamma)}
    c(x,y) - \psi(y)\\
              & =
    \inf_{y \in (\pi_y)(\Gamma)} c(x,y)
    + \inf \{
    -c(x_n,y)+... +
    +c(x_1,y_0)
    -c(x_0,y_0)) :                        \\
              & \hspace{9em}
    n\in \mathbb N, (x_i,y_i) \in \Gamma \
    \forall i = 1,...,n, y_n = y
    \}                                    \\
              & = \phi(x)
  \end{align*}

  Hence, $\phi(x)$ is $c$-concave, and $\phi(x)$ is not constantly equal to $-\infty$,
  since for $x=x_0$, we have
  \begin{align*}
     & c(x_0, y_n) + (\sum_{i=0}^{n-1} c(x_{i+1},y_{i}) ) - \sum_{i=0}^n c(x_i,y_i) \geq 0                               \\
     & \implies \phi(x_0) = \inf \{c(x_0, y_n) + (\sum_{i=0}^{n-1} c(x_{i+1},y_{i}) ) - \sum_{i=0}^n c(x_i,y_i)\} \geq 0
  \end{align*}
  Note that the inequality above is true due to the fact that $\Gamma$ is $c$-CM.

  Now, the only thing left to prove is that $\phi(x)+\phi^c(y) = c(x,y)$ for every $(x,y) \in \Gamma$.
  First, note that for $\epsilon > 0$ and $(x,y) \in \Gamma$, then:
  \begin{align*}
    &\phi(x) = \psi^c(x) = \inf_y c(x,y) - \psi(y) =
    \inf_{y \in (\pi_y)(\Gamma)} c(x,y) - \psi(y)
    \implies \\
    &\exists \bar y \in (\pi_y)(\Gamma) \ : \
    \phi(x)+ \epsilon > c(x,\bar y) - \psi(\bar y)
  \end{align*}
  Also, note that from the definition of $\psi$, we have:
  \begin{align*}
    -\psi(y) \leq -c(x,y) + c(x,\bar y) - c(\bar x_n, \bar y) + ... - c(\bar x_0, \bar y_0)
    \ : \forall i, (\bar x_i, \bar y_i) \in \Gamma
  \end{align*}
  Since this is true for any chain on $\Gamma$ starting on $\bar y$, it's true for the infimum, therefore:
  \begin{equation*}
    -\psi(y) \leq -c(x,y) + c(x,\bar y) - \psi(\bar y) \leq -c(x,y) + \phi(x) + \epsilon
  \end{equation*}

  Since the $\epsilon$ was arbitrary, we can conclude that
  $c(x,y) \leq \phi(x,y) + \psi(x)$. But, we also know that
  \begin{align*}
    \phi^c(y) = \psi^{c c}(y) & = \inf_x c(x,y) - \phi(x)                 \\
                              & = \inf_x c(x,y) - \inf_y c(x,y) - \psi(y) \\
                              & \geq \inf_x c(x,y) - c(x,y) + \psi(y)     \\
                              & = \psi(y)
  \end{align*}
  Hence, $\phi(x)+\phi^c(y) \geq \phi(x) + \psi(y) \geq c(x,y)$.

  Lastly, one would need to show that this $\phi$ is indeed measurable. The general proof is complicated,
  but, if we assume that $c$ is uniformly continuous, then, we know that $c$-transforms are continuous (this was
  shown in Theorem \ref{thm:c-conc}). Since $\phi = \psi^c$, then, $\phi$ is continuous, therefore, it is measurable
  if we consider the Borel $\sigma$-algebra.
\end{prf}

\begin{theorem} (Santambrogio 1.38)
  If $\gamma$ is an optimal transport plan for cost $c$ continuous,
  then $\text{spt } \gamma$ is $c$-CM.
  \label{thm:gamma-cCM}
\end{theorem}
\begin{prf}
  The proof consists in supposing that $\text{spt } \gamma$ is not $c$-CM. Then, we construct a
  $\tilde \gamma \in \Pi(\mu,\nu)$ such that $\int_{X\times Y} c(x,y) \ d\tilde\gamma <
    \int_{X \times Y} c(x,y) \ d\gamma$, which contradicts the optimality of $\gamma$.

  Check \citet{santambrogio2015optimal} for the complete proof.
\end{prf}

\vspace{5mm}
With these results, we can prove the first Strong Duality theorem.

\begin{theorem}
  For $X$ and $Y$ compact metric spaces, and $c:X \times Y \to
    \overline{\mathbb R}$ continuous. Then, $\max\mathrm{(DP)} = \mathrm{\min(KP)}$,
    and DP admits a solution $(\phi,\phi^c)$.
  \label{thm:compactstrongduality}
\end{theorem}
\begin{prf}
  Using Theorem \ref{thm:Santambrogio1.4}, we obtain that $\exists \gamma \in \Pi(\mu,\nu)$
  such that it minimizes the Kantorovich Problem, therefore, by Theorem \ref{thm:gamma-cCM},
  $\text{spt}\gamma$ is $c$-CM.

  By Proposition \ref{thm:c-conc}, we know that a solution to the Dual Problem
  can be found in the set of $c$-concave functions.
  Using \ref{thm:existsPhic}, we can assert that there is a set of $c$-concave
  functions such that $\phi(x)+\phi^c(y) = c(x,y)$ for every $(x,y) \in \text{spt }\gamma$.
  Since $X\times Y$ is compact, then $c$ is uniformly compact, which implies that
  $\phi$ and $\phi^c$ are continuous and bounded.

  Hence, since we already know that $\mathrm{\max(DP)} \leq \mathrm{\min(KP)}$, we conclude that
  $\mathrm{\max(DP)} = \mathrm{\min(KP)}$.
\end{prf}

\begin{theorem}
  For $X$ and $Y$ Polish spaces and $c:X\times Y \to \mathbb R$ uniformly continuous and bounded. Then,
  (DP) admits a solution $(\phi,\phi^c)$ and $\mathrm{\max(DP)}=\mathrm{\min (KP)}$.
  \label{thm:polishStrongDuality}
\end{theorem}

\begin{prf}
  First, note that since $X$ and $Y$ are Polish and $c$ is continuous,
  one can use Theorem \ref{thm:existanceKPpolish} and affirm that exists an optimal solution
  $\gamma$ to (KP).

  By the same arguments used on the proof of Theorem \ref{thm:compactstrongduality},
  we stablish that $\text{spt } \gamma$ is $c$-CM, and that $\phi, \phi^c$ are continuous functions
  such that $\forall (x,y) \in \text{spt } \gamma$, $\phi(x) + \phi^c(y) = c(x,y)$.

  In the Dual Problem, the admissible functions $\phi$ and $\psi$ must be continuous and bounded. Hence,
  we just need to prove that the $\phi$ and $\phi^c$ are indeed bounded. Note that, since $c$ is bounded,
  then, $|c| \leq M \in \mathbb R$ and
  \begin{equation*}
    \phi^c(y) = \inf_x c(x,y) - \phi(x) \leq  \inf_x M - \phi(x) =
    M - \sup_x \phi(x)
  \end{equation*}
  Note that in $\ref{thm:existsPhic}$, we showed that $\phi$ is not constantly $-\infty$. Therefore,
  \begin{equation*}
    -\infty < L < \sup_x \phi(x) \implies
    \phi^c(y) \leq M - \sup_x \phi(x) \leq M - L
  \end{equation*}
  Similarly, since $\phi = \psi^c$ and $\phi^c(y)\geq \psi(y)$ (shown in \ref{thm:gamma-cCM}), then:
  \begin{align*}
    \phi(x) = \inf_y c(x,y) - \psi(y) \geq - M - \sup_y \psi(y) & \geq - M - \sup_y \phi^c(y) \\
                                                                & \geq -M - M + L
  \end{align*}

  Hence, we obtained an upper bound for $\phi^c$ and a lower bound for $\phi$. Now, we obtain an upper bound
  for $\phi$ and a lower bound for $\phi^c$ using a similar argument and relying on the fact that
  $\sup \psi(y) > L > -\infty$:
  \begin{align*}
    \phi(x)  & = \inf_y c(x,y) - \psi(y) \leq M - \sup_y \psi(y) \leq M - L        \\
    \phi^c(x) & = \inf_x c(x,y) - \phi(x) \geq - M - \sup_x \phi(x) \geq -M - M - L
  \end{align*}

  Finally, using the same arguments as Theorem \ref{thm:compactstrongduality}, we conclude that
  $\mathrm{\max (DP)} = \mathrm{\min (KP)}$ and that $(\phi,\phi^c)$ are a solution for the Dual Problem.
\end{prf}

One cost that is of special interest is the quadratic cost $\frac{1}{2} |x-y|^2$. Note that
this cost is neither bounded nor uniformly continuous for non-compact metric spaces. Hence, the previous
theorems do not address it. But one can still prove that Strong Duality is true for such case.

\begin{theorem}
  Let $\mu, \nu \in \mathcal P (\mathbb R^d)$, with $c(x,y) = \frac{1}{2} |x-y|^2$. Suppose that
  $\int|x|^2 d\mu, \int|y|^2 d\nu < +\infty$
  \footnote{This is Theorem 1.40 in \citet{santambrogio2015optimal}, but note that there is a small typo in the book,
    where it states $\int|x|^2 dx, \int|y|^2 dy < + \infty$ instead of the correct $\int|x|^2 d\mu, \int|y|^2 d\nu < +\infty$.}
  . Instead of the original Dual Problem, consider the
  following formulation:
  \begin{flalign}
    \mathrm{(DP')} &&
    \sup \left \{
    \int_{\mathbb R^d} \phi \ d\mu + \int_{\mathbb R^d} \psi \ d\nu \ :
    \phi \in L ^1(\mu) \ , \psi \in L ^1(\nu) \ ,
    \ \phi \oplus \psi \leq c
    \right \}
    &&
    \label{eqt:dualproblemvar}
  \end{flalign}
  Therefore, (DP') admits a solution $(\phi,\psi)$ and $\mathrm{\max (DP')} = \mathrm{\min (KP)}$.
\end{theorem}
\begin{prf}
  First, in the same way as the proof of Theorem \ref{thm:polishStrongDuality}, (KP) has an optimal solution $\gamma$
  with $\text{spt }\gamma$ that is $c$-CM and $ \forall (x,y) \in \text{spt } \gamma$ we have $\phi(x)+ \psi(y) = c(x,y)$.
  We also have that $-\psi(y)=-\phi^c(y)=\sup_x - \frac{|x-y|^2}{2} + \phi(x)$.
  Note that, for $h(x) := \frac{|x|^2}{2} -\phi(x)$
  \begin{align*}
    h^*(y):=\sup_x \langle x,y \rangle - h(x) =
    \sup_x \langle x,y \rangle - \frac{|x|^2}{2}  + \phi(x) &= \\
    \frac{|y|^2}{2} + \sup_x -\frac{|x-y|^2}{2} + \phi(x) = \frac{|y|^2}{2} - \psi(y)
  \end{align*}
  Therefore, $h(x)$ is equal to the Legendre-Fenchel transform of
  $\frac{|y|^2}{2} + \psi(y)$, which implies that $h$ is convex l.s.c. The same argument can be used
  to show that $\frac{|y|^2}{2} - \psi(y)$ is also convex l.s.c.

  Since $\frac{|x^2|}{2} - \phi(x)$ is convex, there exists a supporting hyperplane, hence, it
  is bounded from below by a linear function, which implies that
  \begin{align*}
    \frac{|x^2|}{2} - \phi(x) \geq \alpha \langle x,y \rangle  + \beta & \implies
    \phi(x) \leq \frac{|x^2|}{2} - \alpha \langle x,y \rangle - \beta             \\
                                                                       & \implies
    \int_{\mathbb R^d} \phi(x) \ d\mu \leq \int_{\mathbb R^d} \frac{|x^2|}{2} - \alpha \langle x,y \rangle - \beta \ d\mu < +\infty
  \end{align*}

  The same argument can be made for $\psi$, which means that $\phi_+ \in  L^1(\mu)$ and $\psi_+ \in L^1(\nu)$.
  Due to the fact that $\phi(x) + \psi(y) = c(x,y)$ in the support of $\gamma$, then
  \begin{equation*}
    \int_{\mathbb R^d \times \mathbb R^d} \phi \oplus \psi \ d\gamma  =
    \int_{\mathbb R^d \times \mathbb R^d} c \ d\gamma  \geq 0
  \end{equation*}
  Which implies that the negative portions of $\phi$ and $\psi$ are also integrable, leading us to conclude
  that $\phi \in L^1(\mu)$ and $\psi \in L^1(\nu)$.

  Finally, by the same arguments as the previous theorems, we prove that
  $\mathrm{\max(DP')}= \mathrm{\min(KP)}$.

\end{prf}

\vspace{5mm}
A stronger result can be proven regarding the duality of KP. We'll present it here without a proof.
\begin{theorem}(Santambrogio 1.42)
  For $X$ and $Y$ Polish spaces and $c:X\times Y \to \mathbb R\cup \{+\infty\}$ l.s.c and bounded from below. Then,
  $\mathrm{\sup(DP)}=\mathrm{\min (KP)}$.
  Note that in this theorem, one cannot guarantee the existence of the $(\phi,\psi)$ that maximize the Dual Problem.
  \label{thm:strongerDuality}
\end{theorem}


If the cost $c(x,y)$ is actually a distance metric (Def. \ref{def:metric}),
then we can prove the following result:
\begin{theorem}
  Let $X$ be a metric space, and $c:X \times X \to \mathbb{R}$, where $c$ is a distance metric. Therefore,
  a function $f:X \to \mathbb{R}$ is $c$-concave if and only if it is Lipschitz continuous with a constant
  less than 1 with respect to the distance $c$.
  We call $\text{Lip}_1^{(c)}$ this set of Lipschitz functions with constant less than 1. Moreover,
  $f^c = -f$.
  \label{thm:cConcaveLip1}
\end{theorem}
\begin{prf}

  $\implies$) Let $f:X \to \mathbb R$ be a $c$-concave function. Hence, $\exists \ g:X \to \overline{\mathbb R}$ such that
  \begin{equation*}
    f(x) := \inf_y c(x,y) - g(y)
  \end{equation*}
  Using the triangle inequality of the cost, we get:
  \begin{gather*}
    c(x,y) \leq c(x,z) + c(z,y) \implies \sup_y c(x,y) - c(y,z) \leq c(x,z) \\
    c(y,z) \leq c(y,x) + c(x,z) \implies \sup_y c(y,z) - c(x,y) \leq c(x,z) \\
    \therefore \\
    \sup_y |c(y,z) - c(x,y)| \leq c(x,z)
  \end{gather*}
  Therefore,
  \begin{align*}
    |f(x) - f(z)| & = |\inf_y \{c(x,y) - g(y) \} \ - \ \inf_y \{c(z,y) - g(y)\}| \leq \\
                  & \underset{\ref{lem:infsup_ineq}}{\leq}
    \sup_y |c(x,y) - c(z,y)| \leq c(x,z)
  \end{align*}

  $\impliedby$) Let $f \in \text{Lip}^{(c)}_1$. Using the Lipschitz inequality,
  \begin{equation*}
    f(x) - f(y) \leq c(x,y) \implies f(x) \leq \inf_y c(x,y) + f(y)
  \end{equation*}
  But note that $f(x) = c(x,x) + f(x) \geq \inf_y c(x,y) - f(y)$. This implies that
  $f(x) = \inf_y c(x,y) + f(y)$. Hence, $f(x) = g^c(x)$, where $g(y) = -f(y)$. Which proves
  that $f$ is $c$-concave, and $f = (-f)^c$. Finally, note that $-f$ is also $\text{Lip}_1$,
  therefore, the same argumentation leads to $-f = f^c$.
\end{prf}
\vspace{5mm}

Lastly, using  Theorems
\ref{thm:strongerDuality} and \ref{thm:cConcaveLip1}, one obtains the famous
Kantorovich-Rubinstein Duality:

\begin{theorem}(Kantorovich-Rubinstein)

  Let $(X,d)$ be a Polish space with metric $d$, and cost function $c(x,y) = d(x,y)$.
  Then, for $\mu, \nu \in \mathcal P(X)$, the Kantorovich Problem
  is equivalent to
  \begin{equation}
      \sup \left \{
      \int_X \phi \ d\mu - \int_X \phi \ d\nu \ :
      \phi \in Lip_1(X)
      \right \}
  \end{equation}
  \label{thm:Kantorovich-Rubinstein}
\end{theorem}
