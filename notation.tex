
\newpage
\chapter*{Notation}
\addcontentsline{toc}{chapter}{Notation}

The following symbols are used in the text without always recalling their meaning.

\begin{itemize}
	\item $\mathcal M(X), \mathcal M_+(X)$: Space of finite measures and finite positive measures on $X$, respectively.
	\item $\mathcal P(X), \mathcal P_p(X)$: Space of probability measures and space of probability
	measures with $p$th finite moment, respectively.
	\item $\mathbbm 1_A(x)$: Indicator function of set A, i.e. $\mathbbm 1_A(x)=1$ if $x \in A$ and 0 otherwise.
	\item $\mathbf 1_n$: $n$ dimensional vector of ones.
	\item $id$: Identity operator, i.e. $id(x)=x$.
	\item $\oplus$: For $\phi: X \to \mathbb R$, $\psi: Y \to \mathbb R$, then $(\phi \oplus \psi)(x,y) = \phi(x) + \psi(y)$.
	\item $\pi_X$: Projection operator on $X$, i.e. for $\pi_X : X \times Y \to X$, then $\pi_X(x,y)= x, \ \forall (x,y) \in X \times Y$.
	\item $\Pi(\mu,\nu)$: Coupling of measures $\mu$ and $\nu$.
	\item $\mathbb R_+$: Positive real number greater or equal than 0.
	\item $\mathbb{\overline R}$: Real numbers extended to include $+\infty$ and $-\infty$.
	\item $C(X)$: Set of functions $f: X \to \mathbb R$, where $f$ is continuous.
	\item $C_b(X)$: Set of functions $f: X \to \mathbb R$, where $f$ is continuous and bounded.
	\item $C_0(X)$: Set of functions $f: X \to \mathbb R$, where $f$ is continuous and goes to zero at infinity.
	\item $C_c(X)$: Set of functions $f: X \to \mathbb R$, where $f$ is continuous and has compact support.
	\item $\mu_n \rightharpoonup \mu$: Measure $\mu_n$ converges weakly to $\mu$.
	\item $OT_c(\mu,\nu)$: Optimal Transport cost between measures $\mu$ and $\nu$ for a ground cost function $c$.
	\item $OT_{c,\varepsilon}(\mu,\nu)$, $\overline{OT}_{c,\varepsilon}(\mu,\nu)$:
	Entropic Optimal Transport distance and the Entropic Optimal Transport cost
	between measures $\mu$ and $\nu$ for a ground cost function $c$, respectively.
	\item $W_p$, $W_{p,\varepsilon}$, $S_{c,\varepsilon}$, $SW$, $GW$: Wasserstein distance, Entropic Wasserstein distance,
	Sinkhorn divergence,
	Sliced-Wasserstein distance and Gromov-Wasserstein distance, respectively.
	\item $\mathrm{KL}$: Kullback-Leibler divergence.
\end{itemize}

