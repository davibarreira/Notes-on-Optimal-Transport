\documentclass{amsart}

%\usepackage{authblk}
\usepackage{amsthm,amsmath,amssymb,epsfig,graphicx,mathtools}
\usepackage{hyperref}
\usepackage{dsfont}
\usepackage{color}
\usepackage{enumerate}
\usepackage{url}
%`\usepackage{comment}r
%\usepackage[ruled]{algorithm2e}
%\usepackage{algpseudocode}
\usepackage[table]{xcolor}
\usepackage[export]{adjustbox}
%---------------------------------------------------------


%\usepackage{showkeys}
\usepackage[ulem=normalem,draft]{changes}
%\usepackage{easyReview}
\usepackage{cancel}
\newcommand{\joao}{\textcolor{cyan}}


\makeindex

%%%%%%%%%%%%%%%%%%%%%%%%%%%%%%%%%%%%%%%%%%%%%%%%%%%%%%%%%
%               PAGE FORMAT
\if {
\hoffset=-1in
\voffset=-1in
\topmargin=2,75cm
\headsep=1cm
\oddsidemargin=2cm
\evensidemargin=2cm
\textwidth=13cm
\textheight=18,5cm
\tolerance=2000
} \fi

\if{
%\hoffset=-0.5in
\voffset=-1in
\topmargin=2,75cm
\headsep=1cm
\oddsidemargin=1cm
\evensidemargin=1cm
\textwidth=16cm
\textheight=20,5cm
\tolerance=2500
}\fi

\numberwithin{equation}{section}
%%%%%%%%%%%%%%%%%%%%%%%%%%%%%%%%%%%%%%%%%%%%%%%%%%%%%%%%%
%           FIGURES DIRECTORY %  \newcommand\Figdir{figures/}

%%%%%%%%%%%%%%%%%%%%%%%%%%%%%%%%%%%%%%%%%%%%%%%%%%%%%%%%%
%           THEOREMS
\newtheorem{theorem}{Theorem}[section]
\newtheorem{lemma}[theorem]{Lemma}
\newtheorem{proposition}[theorem]{Proposition}
\newtheorem{corollary}[theorem]{Corollary}
\newtheorem{assumption}{Assumption}
\newtheorem{claim}{Claim}

\theoremstyle{definition}
\newtheorem{definition}{Definition}[section]

\theoremstyle{remark}
\newtheorem{remark}[theorem]{Remark}
%%%%%%%%%%%%%%%%%%%%%%%%%%%%%%%%%%%%%%%%%%%%%%%%%%%%%%%%%
%             MACROS
%%%%%%%%%%%%%%%%%%%%%%%%%%%%%%%%%%%%%%%%%%%%%%%%%%%%%%%%%%%%
%                 TEX MACRO FILE, J.M. MACHADO
%                       DECEMBER 5, 2017
%%%%%%%%%%%%%%%%%%%%%%%%%%%%%%%%%%%%%%%%%%%%%%%%%%%%%%%%%%%%



\newcommand{\newpagec}{}

\newcommand{\XXX}[1]{{\bf \; XXX #1}}
%\newcommand{\XXX}[1]{}

%%%--------------------------------------

\def\densesubset{\mathop{\stackrel{\rightharpoonup}{\subset}}}

\def\dd{{\rm d}}
\newcommand{\ddt}{\frac{\rm d}{{\rm d} t} }
\newcommand{\dotex}[1]{{\frac{\displaystyle d #1}{\displaystyle dt}}}
\newcommand{\ideal}[1]{\langle #1 \rangle}

%%%%%%%%%%%%%%%%%%%%%%%%%%%%%%%%%%%%%%% EMPHAZISE, ETC 
\newcommand{\new}[1]{{\em #1}}
\newcommand{\newd}[2]{{\em #1}\index{#2}}
%\newcommand{\prob}[1]{{\em #1}\index{#1}}
\newcommand{\mrm}[1]{\text{\rm #1}}
%%%%%%%%%%%%%%%%%%%%%%%%%%%%%%%%%%%%% GRAPHES 
% \newcommand{\arc}[2]{(#1,#2)}
\def\arcset{{\mathcal A}}
\def\nodeset{{\mathcal N}}
\def\incid{{M}}
\newcommand{\edge}[2]{\{#1,#2\}}
%%%%%%%%%%%%%%%%%%%%%%%%%%%%%%%%%%%%% DIVERS S. Gaubert

\newcommand{\ov}[1]{\overline{#1}}
\newcommand{\arcm}[2]{#1#2}

\newcommand{\diagdots}{_{^{\big\cdot}\cdot _{\big\cdot}}}

\newcommand{\transp}{^\top}
\newcommand{\wglobal}[1]{\typeout{warning: #1 global definition}#1}
\newcommand{\promis}{\typeout{warning: verifier que c est fait}}
\newcommand{\ext}{\operatorname{extr}}
\newcommand{\co}{\operatorname{co}}
\newcommand{\cob}{\operatorname{\overline{co}}}
\newcommand{\set}[2]{\{#1\mid\,#2\}}

\newenvironment{sgdelicat}{\small}{}
\newenvironment{sgremarque}{\begin{remark}}{\hfill $\bullet$\end{remark}}

\def\weight(#1,#2){c_{#1,#2}}
%%%%%%%%%%%%%%%%%%%%%%%%%%%%%%%%%%%%% DIVERS RO
\def\Kh{\hat{K}} 
\def\Eb{\bar{E}}
\def\Adj{{\rm Adj}}
\def\Adjc{{\rm Adjc}}
\def\deg{{\rm deg}}
\def\red{{\rm red}}
\def\ra{{\rm ra}}
\def\ovdeg{{\overline{\rm deg}}}
\def\degu{\underline{\rm deg}}
\def\Mh {{\hat M}}
%%%%%%%%%%%%%%%%%%%%%%%%%%% Prog Dyn
\def\qbo {{\bf q}}
\def\ubo {{\bf u}}
\def\ybo {{\bf y}}
\def\uv {{\underline v}}


\newcommand{\norm}[1]{\left\lVert#1\right\rVert}
\newcommand{\inner}[1]{\left\langle#1\right\rangle}
\newcommand{\partialderivative}[2]{\frac{\partial#1}{\partial#2}}
%\newcommand{\thm}{th\'eor\`eme }
\newcommand{\un}{un }
% MATHS
\newcommand\meas{\mathop{\rm meas}}
%\renewcommand\rank{\mathop{\rm rang}}
\newcommand\mathpi{\mathop\Pi}

% MATHS
  \renewcommand\meas{\mathop{\rm meas}}  
%  \renewcommand\rank{\mathop{\rm rank}}
%} \fi
%%%%%%%%%%%%%%%%%%%%%%%%%%%%%%%%%%%%%%%%%%%%%%%%%%%%%%%%%%

\if { 
\renewcommand{\chapterl}[1]{\chapter{\LARGE \bf {#1}}\vspace{-40mm}}
\renewcommand{\sectionl}[1]{\section{\LARGE \bf {#1}}}
\renewcommand{\subsectionl}[1]{\subsection{\LARGE \bf {#1}}}
\renewcommand{\subsubsectionl}[1]{\subsubsection{\Large \bf {#1}}}
\renewcommand{\paragraphl}[1]{\paragraph{\LARGE \bf {#1}}}
\renewcommand{\subparagraphl}[1]{\subparagraph{\LARGE \bf {#1}}}

\renewcommand{\captionl}[1]{\caption{\LARGE \bf {#1}}}

\renewcommand{\Largel}{\Large}}{{\caption}}
} \fi

%%%%%%%%%%%%%%%%%%%%%%%%%%%%%%%%%%%%%%%%%%%%%%%%%%%%%%%%%%%%
\newcommand\findem{\hfill{$\blacksquare$}\medskip}
%%%%%%%%%%%%%%%%%%%%%%%%%%%%%%%%%%%%%%%%%%%%%%%%%%%%%%%%%%%%


%%%%%%%%%%%%%%%%%%%%%%%%%% PROOFS %%%%%%%%%%%%%%%%%%%%%%%%%%%%
% \newcommand{\debdem}{\begin{proof}} \def\findem{\end{proof}}

%%% \newenvironment{proof}{{\noindent \bf D\'emonstration.}\quad}{\hfill{$\blacksquare$}\medskip}
\newenvironment{proofdu}[1]{{\noindent \bf D\'emonstration #1.}\quad}{\hfill{$\blacksquare$}\medskip}

%%% \newcommand{\qed}{\findem}
%%%%%%%%%%%%%%%%%%%%%%%%%%%%%%%%%%%%%%%%%%%%%%%%%%%%%%%%%%%%

\newcommand\ML{{\rm M}}

%%%%%%%%%%%%%%%%%%%%%%%%%%%%%%%% CHECK CHECH %%%%%%%%%%%%%%%%%%%%%%%%
\def\ack{\check{a}}
\def\bck{\check{b}}
\def\cck{\check{c}}
\def\dck{\check{d}}
\def\eck{\check{e}}
\def\fck{\check{f}}
\def\gck{\check{g}}
\def\hck{\check{h}}
\def\ick{\check{i}}
\def\jck{\check{j}}
\def\kck{\check{k}}
\def\lck{\check{l}}
\def\mck{\check{m}}
\def\nck{\check{n}}
\def\ock{\check{o}}
\def\pck{\check{p}}
\def\qck{\check{q}}
\def\rck{\check{r}}
\def\sck{\check{s}}
\def\tck{\check{t}}
\def\uck{\check{u}}
\def\vck{\check{v}}
\def\wck{\check{w}}
\def\xck{\check{x}}
\def\yck{\check{y}}
\def\zck{\check{z}}

%%%%%%%%%%%%%%%%%%%%%%%%%%%%%%%% HAT %%%%%%%%%%%%%%%%%%%%%%%%
\def\ah{\hat{a}}
\def\bh{\hat{b}}
\def\ch{\hat{c}}
\def\dh{\hat{d}}
\def\eh{\hat{e}}
\def\fh{\hat{f}}
\def\gh{\hat{g}}
\def\hh{\hat{h}}
\def\ih{\hat{i}}
\def\jh{\hat{j}}
\def\kh{\hat{k}}
\def\lh{\hat{l}}
\def\mh{\hat{m}}
\def\nh{\hat{n}}
\def\oh{\hat{o}}
\def\ph{\hat{p}}
\def\qh{\hat{q}}
\def\rh{\hat{r}}
\def\sh{\hat{s}}
\def\th{\hat{t}}
\newcommand\tth{\hat{t}}
\def\uh{\hat{u}}
\def\vh{\hat{v}}
\def\wh{\hat{w}}
\def\xh{\hat{x}}
\def\yh{\hat{y}}
\def\zh{\hat{z}}
%%%%%%%%%%%%%% UPPERCASE HAT %%%%%%%%
\newcommand\Ah{\hat{A}}
\newcommand\Bh{\hat{B}}
\newcommand\Ch{\hat{C}}
\newcommand\Dh{\hat{D}}
\newcommand\Eh{\hat{E}}
\newcommand\Hh{\hat{H}}
\newcommand\Ih{\hat{I}}
\newcommand\Lh{\hat{L}}
\newcommand\Ph{\hat{P}}
\newcommand\Qh{\hat{Q}}
\newcommand\Th{\hat{T}}
\newcommand\Uh{\hat{U}}
\newcommand\Vh{\hat{V}}
\newcommand\Wh{\hat{W}}
%%%%%%%%%%%%%%%%%%%%%%%%%%%%%%%% BAR %%%%%%%%
\def\ab{\bar{a}}
\def\bb{\bar{b}}
\def\cb{\bar{c}}
\def\db{\bar{d}}
\def\eb{\bar{e}}
\def\fb{\bar{f}}
\def\gb{\bar{g}}
\def\hb{\bar{h}}
\def\ib{\bar{i}}
\def\jb{\bar{j}}
\def\kb{\bar{k}}
\def\lb{\bar{l}}
\def\mb{\bar{m}}
\def\nb{\bar{n}}
\def\ob{\bar{o}}
\def\pb{\bar{p}}
\def\qb{\bar{q}}
\def\rb{\bar{r}}
\def\sb{\bar{s}}
\def\tb{\bar{t}}
\def\ub{\bar{u}}
\def\vb{\bar{v}}
\def\wb{\bar{w}}
\def\xb{\bar{x}}
\def\yb{\bar{y}}
\def\zb{\bar{z}} 

%%%%%%%%%%%%%%%%%%%%%%%%%%%%%%%% UPPERCASE BAR %%%%%%%%%%%%%%
\def\Ab{\bar{A}}
\def\Bb{\bar{B}}
\def\Cb{\bar{C}}
\def\Db{\bar{D}}
\def\Eb{\bar{E}}
\def\Fb{\bar{F}}
\def\Gb{\bar{G}}
\def\Hb{\bar{H}}
\def\Ib{\bar{I}}
\def\Jb{\bar{J}}
\def\Kb{\bar{K}}
\def\Lb{\bar{L}}
\def\Mb{\bar{M}}
\def\Nb{\bar{N}}
\def\Ob{\bar{O}}
\def\Pb{\bar{P}}
\def\Qb{\bar{Q}}
\def\Rb{\bar{R}}
\def\Sb{\bar{S}}
\def\Tb{\bar{T}}
\def\Ub{\bar{U}}
\def\Vb{\bar{V}}
\def\Wb{\bar{W}}
\def\Xb{\bar{X}}
\def\Yb{\bar{Y}}
\def\Zb{\bar{Z}}

\def\Cbar{\bar{C}}
\def\Tbar{\bar{T}}
\def\Zbar{\bar{Z}}
%%%%%%%%%%%%%%%%%%%%%%%%%%%%%%%% TILDE %%%%%%%%%%%%%%%%%%%%%%%%

\def\at{\tilde{a}}
\def\bt{\tilde{b}}
\def\ct{\tilde{c}}
\def\dt{\tilde{d}}
\def\et{\tilde{e}}
\def\ft{\tilde{f}}
\def\gt{\tilde{g}}
\def\hit{\tilde{h}}
\def\iit{\tilde{i}}
\def\jt{\tilde{j}}
\def\kt{\tilde{k}}
\def\lt{\tilde{l}}
\def\mt{\tilde{m}}
\def\nt{\tilde{n}}
\def\ot{\tilde{o}}
\def\pt{\tilde{p}}
\def\qt{\tilde{q}}
\def\rt{\tilde{r}}
\def\st{\tilde{s}}
\def\tt{\tilde{t}}
\def\ut{\tilde{u}}
\def\vt{\tilde{v}}
\def\wt{\tilde{w}}
\def\xt{\tilde{x}}
\def\yt{\tilde{y}}
\def\zt{\tilde{z}}
%%%%%%%%%%%%%%%%%%%%%%%%%%%%%%%% UPPERCASE TILDE %%%%%%%%%%%%%%%%
\def\At{\tilde{A}} 
\def\Lt{\tilde{L}} 
\def\Pt{\tilde{P}} 
\def\Qt{\tilde{Q}} 
\def\Rt{\tilde{R}} 
\def\St{\tilde{S}} 
\def\Vt{\tilde{V}} 
\def\Wt{\tilde{W}} 
\def\Xt{\tilde{X}} 
\def\Yt{\tilde{Y}} 
\def\Zt{\tilde{Z}} 


\def\tA{\tilde{A}} 
\def\tL{\tilde{L}} 
\def\tS{\tilde{S}} 
\def\tV{\tilde{V}} 
\def\tW{\tilde{W}} 
\def\tX{\tilde{X}} 
\def\tY{\tilde{Y}} 
\def\tZ{\tilde{Z}} 


%%%%%%%%%%%%%%%%%%%%%%%%%%%%%%%% BOLDFACE %%%%%%%%%%%%%%%%%%%%%%%
\def\ebf { {\bf e}}  
\def\fbf { {\bf f}}  
\def\gbf { {\bf g}}  
\def\hbf { {\bf h}}  
\def\mbf { {\bf m}}  
\def\pbf { {\bf p}}  
\def\qbf { {\bf q}}  
\def\rbf { {\bf r}}  
\def\sbf { {\bf s}}  
\def\ubf { {\bf u}}  
\def\vbf { {\bf v}}  
\def\wbf { {\bf w}}  
\def\xbf { {\bf x}}  
\def\ybf { {\bf y}}  
\def\zbf { {\bf z}}  
%%%%%%%%%%%%%%%%%%%%%%%%%%%%%%%% HAT BOLDFACE %%%%%%%%%%%%%%%%%%%%%%%
\def\hebf {\hat {{\bf e} }}  
\def\hfbf {\hat {{\bf f} }}  
\def\hgbf {\hat {{\bf g} }}  
\def\hhbf {\hat {{\bf h} }}  
\def\hpbf {\hat {{\bf p} }}
\def\hqbf {\hat {{\bf q} }}  
\def\hrbf {\hat {{\bf r} }}  
\def\hsbf {\hat {{\bf s} }}  
\def\hubf {\hat {{\bf u} }}  
\def\hvbf {\hat {{\bf v} }}  
\def\hwbf {\hat {{\bf w} }}  
\def\hxbf {\hat {{\bf x} }}  
\def\hybf {\hat {{\bf y} }}  
\def\hzbf {\hat {{\bf z} }}  
%%%%%%%%%%%%%%%%%%%%%%%%%%%%%%%% BAR BOLDFACE %%%%%%%%%%%%%%%%%%%%%%%
\def\bebf {\bar {{\bf e} }}  
\def\bfbf {\bar {{\bf f} }}  
\def\bgbf {\bar {{\bf g} }}  
\def\bhbf {\bar {{\bf h} }}  
\def\bpbf {\bar {{\bf p} }}
\def\bqbf {\bar {{\bf q} }}  
\def\brbf {\bar {{\bf r} }}  
\def\bsbf {\bar {{\bf s} }}  
\def\bubf {\bar {{\bf u} }}  
\def\bvbf {\bar {{\bf v} }}  
\def\bwbf {\bar {{\bf w} }}  
\def\bxbf {\bar {{\bf x} }}  
\def\bybf {\bar {{\bf y} }}  
\def\bzbf {\bar {{\bf z} }}  
%%%%%%%%%%%%%%%%%%%%%%%%%%%%%%%% TILDE BOLDFACE %%%%%%%%%%%%%%%%%%%%%%%
\def\tebf {\tilde {{\bf e} }}  
\def\tfbf {\tilde {{\bf f} }}  
\def\tgbf {\tilde {{\bf g} }}  
\def\thbf {\tilde {{\bf h} }}  
\def\tpbf {\tilde {{\bf p} }}
\def\tqbf {\tilde {{\bf q} }}  
\def\trbf {\tilde {{\bf r} }}  
\def\tsbf {\tilde {{\bf s} }}  
\def\tubf {\tilde {{\bf u} }}  
\def\tvbf {\tilde {{\bf v} }}  
\def\twbf {\tilde {{\bf w} }}  
\def\txbf {\tilde {{\bf x} }}  
\def\tybf {\tilde {{\bf y} }}  
\def\tzbf {\tilde {{\bf z} }}  

%%%%%%%%%%%%%%%%%%%%%%%%%%%%%%%% CAL + TILDE  %%%%%%%%%%%%%%%%%%%%%
\def\calat{{\tilde {\mathcal A}}}

%%%%%%%%%%%%%%%%%%%%%%%%%%%%%%%% CAL %%%%%%%%%%%%%%%%%%%%%%%%%%%%
\def\cala{{\mathcal  A}}
\def\calb{{\mathcal B}}
\def\calc{{\mathcal C}}
\def\cald{{\mathcal D}}
\def\cale{{\mathcal E}}
\def\calf{{\mathcal F}}
\def\calg{{\mathcal G}}
\def\calh{{\mathcal H}}
\def\cali{{\mathcal I}}
\def\calj{{\mathcal J}}
\def\calk{{\mathcal K}}
\def\call{{\mathcal L}}
\def\calm{{\mathcal M}}
\def\caln{{\mathcal N}}
\def\calo{{\mathcal O}}
\def\calp{{\mathcal P}}
\def\calq{{\mathcal Q}}
\def\calr{{\mathcal R}}
\def\cals{{\mathcal S}}
\def\calt{{\mathcal T}}
\def\calu{{\mathcal U}}
\def\calv{{\mathcal V}}
\def\calw{{\mathcal W}}
\def\calx{{\mathcal X}}
\def\caly{{\mathcal Y}}
\def\calz{{\mathcal Z}}
%%%%%%%%%%%%%%%%%%%%%%%%%%%% OTHER CAL %%%%%%%%%%%%%%%%%%%%%%%%%%
\def\cA{{\mathcal A}}
\def\cF{{\mathcal F}}
\def\cL{{\mathcal L}}
\def\cP{{\mathcal P}}
\def\cQ{{\mathcal Q}}
\def\cR{{\mathcal R}}
\def\cV{{\mathcal V}}
\def\cW{{\mathcal W}}
\def\cO{{\mathcal O}}

\def\cPt{\widetilde{\cal P}}
%%%%%%%%%%%%%%%%%%%%%%%%%%%%%%%% MATHCAL %%%%%%%%%%%%%%%%%%%%%%%%
\def\A{\mathcal{A}}
\def\B{\mathcal{B}}
\def\C{\mathcal{C}}
\def\D{\mathcal{D}}
\def\E{\mathcal{E}}
\def\F{\mathcal{F}}
\def\G{\mathcal{G}}
\def\H{\mathcal{H}}
\def\I{\mathcal{I}}
\def\J{\mathcal{J}}
\def\K{\mathcal{K}}
\def\L{\mathcal{L}}
\def\M{\mathcal{M}}
\def\N{\mathcal{N}}
\def\P{\mathcal{P}}
\def\Q{\mathcal{Q}}
\def\R{\mathcal{R}}
\def\SS{\mathcal{S}}
\def\T{\mathcal{T}}
\def\U{\mathcal{U}}
\def\V{\mathcal{V}}
\def\W{\mathcal{W}}
\def\X{\mathcal{X}}
\def\Y{\mathcal{Y}}
\def\Z{\mathcal{Z}}
%%%%%%%%%%%%%%%%%%%%%%%%%%%% AUTRES MATHCAL %%%%%%%%%%%%%%%%%%
\newcommand{\sB}{\mathcal{B}}
\newcommand{\sC}{\mathcal{C}}
\newcommand{\sE}{\mathcal{E}}
\newcommand{\sL}{\mathcal{L}}
\newcommand{\sS}{\mathcal{S}}
\newcommand{\sT}{\mathcal{T}}
\newcommand{\sV}{\mathcal{V}}
%%%%%%%%%%%%%%%%%%%%%%% OVERLINE  MATHCAL  %%%%%%%%%%%%%%%%%%%%%%%%
\newcommand{\Hbc}{\overline{\mathcal H}}


%%%%%%%%%%%%%%%%%%%%%%% OVERLINE   %%%%%%%%%%%%%%%%%%%%%%%%
\def\HH{\overline{\bf H}}


\newcommand{\Xbo}{\overline{X}}
\newcommand{\Ybo}{\overline{ Y}}

%%%%%%%%%%%%%%%%%%%%%%% UNDERLINE %%%%%%%%%%%%%%%%%%%%%%%%
\def\au{\underline{a}}
\def\xu{\underline{x}}
%
\def\uV{\underline{V}}

%%%%%%%%%%%%%%%%%%%%%%% GREEKS %%%%%%%%%%%%%%%%%%%%%%%%
\def\eps{\varepsilon}
\def\Om{{\Omega}}
\def\Omb{{\Omegab}}
\def\om{{\omega}}
\def\ml{\lambda}
%%%%%%%%%%%%%%%%%%%%%%% GREEK HATS %%%%%%%%%%%%%%%
\def\alphah{\hat{\alpha}}
\def\betah{\hat{\beta}}
\def\ellh{\hat{\ell}}
\def\etah{\hat{\eta}}
\def\gamh{\hat{\gamma}}
\def\kappah{\hat{\kappa}}
\def\mlh{\hat{\lambda}}
\def\lambdah{\hat{\lambda}}
\def\Lambdah{\hat{\Lambda}}
\def\muh{\hat{\mu}}
\def\nuh{\hat{\nu}}
\def\pih{{\hat\pi}}
\def\Phih{{\hat\Phi}}
\def\Phih{{\hat\Phi}}
\def\psih{{\hat\psi}}
\def\Psih{{\hat\Psi}}
\def\Psih{{\hat\Psi}}
\def\rhoh{{\hat\rho}}
\def\sigmah{{\hat\sigma}}
\def\tauh{{\hat\tau}}
\def\thetah{{\hat\theta}}
\def\xih{{\hat\xi}}

%%%%%%%%%%%%%%%%%%%%%%% GREEK BARS %%%%%%%%%%%%%%%
\def\alphab{\bar{\alpha}}
\def\betab{\bar{\beta}}
\def\chib{\bar{\chi}}
\def\gammab{\bar{\gamma}}
\def\deltab{\bar{\delta}}
\def\ellb{\bar{\ell}}
\def\epsb{\bar{\varepsilon}}
\def\etab{\bar{\eta}}
\def\kappab{\bar{\kappa}}
\def\mlb{\bar{\lambda}}
\def\lambdab{\bar{\lambda}}
\def\Lambdab{\bar{\Lambda}}
\def\mub{\bar{\mu}}
\def\nub{\bar{\nu}}
\def\pib{{\bar\pi}}
\def\Phib{{\bar\Phi}}
\def\phib{{\bar\phi}}
\def\psib{{\bar\psi}}
\def\Psib{{\bar\Psi}}
\def\Psih{{\hat\Psi}}
\def\sigmab{{\bar\sigma}}
\def\taub{{\bar\tau}}
\def\thetab{{\bar\theta}}
\def\varphib{{\bar\varphi}}
\def\xib{{\bar\xi}}
\def\zetab{{\bar\zeta}}

%%%%%%%%%%%%%%%%%%%%%%% GREEK TILDE %%%%%%%%%%%%%%%
\def\alphat{\tilde{\alpha}}
\def\betat{\tilde{\beta}}
\def\ellt{\tilde{\ell}}
\def\etat{\tilde{\eta}}
\def\kappat{\tilde{\kappa}}
\def\mlt{\tilde{\lambda}}
\def\lambdat{\tilde{\lambda}}
\def\Lambdat{\tilde{\Lambda}}
\def\mut{\tilde{\mu}}
\def\nut{\tilde{\nu}}
\def\omt{\tilde{\omega}}
\def\Omt{\tilde{\Omega}}
\def\pit{{\tilde\pi}}
\def\Phit{{\tilde\Phi}}
\def\Phit{{\tilde\Phi}}
\def\psit{{\tilde\psi}}
\def\Psit{{\tilde\Psi}}
\def\Psih{{\hat\Psi}}
\def\sigmat{{\tilde\sigma}}
\def\taut{{\tilde\tau}}
\def\thetat{{\tilde\theta}}
\def\thetat{{\tilde\theta}}
\def\xit{{\tilde\xi}}
\def\varphit{{\tilde\varphi}}


%%%%%%%%%%%%%%%%%%GREEK BOLDFACE AND BAR BOLDFACE %%%%%%%%%%
\def\alphabf {\boldsymbol\alpha}
\def\balphabf {\bar{\boldsymbol\alpha}}
\def\betabf {\boldsymbol\beta}
\def\bbetabf {\bar{\boldsymbol\beta}}
\def\etabf {\boldsymbol\eta}
\def\betabf {\bar{\boldsymbol\eta}}
\def\mubf {\boldsymbol\mu}
\def\blambdabf {\bar{\boldsymbol\lambda}}
\def\bmubf {\bar{\boldsymbol\mu}}
\def\pibf {\boldsymbol\pi}
\def\bpibf {\bar{\boldsymbol\pi}}
\def\xibf {\boldsymbol\xi}
\def\bxibf {\bar{\boldsymbol\xi}}
\def\taubf {\boldsymbol\tau}

%%%%%%%%%%%%%%%%%%%%%%%% MATH OPERATORS %%%%%%%%%%%%%%%%
\def\1B{{\bf  1}}

\newcommand{\ceil}[1]{\lceil #1 \rceil}
\newcommand{\floor}[1]{\lfloor #1 \rfloor}

\def\densesubset{\mathop{\stackrel{\rightharpoonup}{\subset}}}

\def\ad{\mathop{\rm ad}}
\def\affhull{\mathop{\rm affhull}}
\def\argmin{\mathop{\rm argmin}}
\def\argmax{\mathop{\rm argmax}}
\def\cl{\mathop{\rm cl}}
\newcommand\Comp{\mathop{\rm Comp}}
\def\caffhull{\mathop{\rm \overline{affhull}}}
\def\cone{\mathop{\rm cone}}
\def\core{\mathop{\rm core}}
\newcommand\conebar{\mathop{\rm \overline{cone}}}
\def\conv{\mathop{\rm conv}}
\def\convbar{\mathop{\rm \overline{conv}}}
\def\deg{\mathop{\rm deg}}
\def\det{\mathop{\rm det}}
\def\dett{\mathop{\rm det}}
\def\diag{{\mathop{\rm diag}}}
\newcommand\diam{\mathop{\rm diam}}
\def\dist{\mathop{\rm dist}}
\def\ddiv{\mathop{\rm div}}
\def\dom{\mathop{{\rm dom}}}
% \def\epi{\mathop{\text{\'epi}}}
\def\epi{\mathop{\text{epi}}}
\newcommand{\ess}{\mathop{\rm ess}}
\def\essinf{\mathop{\rm essinf}}
\def\essup{\mathop{\rm esssup}}
\def\esssup{\mathop{\rm esssup}}
\def\intt{\mathop{\rm int}}
\def\inv{\mathop{\rm inv}}
\def\isom{\mathop{\rm Isom}}
\def\lin{\mathop{\rm lin}}
\newcommand\Jac{{\mathop{\rm Jac}}}
\newcommand\Lip{\mathop{\rm Lip}}
\newcommand\snorm{\|}
\newcommand\lnlt{\mathop{\rm lnlt}}
% \newcommand\psicirc{{\stackrel \circ \psi}}
\newcommand\psicirc{ \psi}
\def\range{\mathop{\rm Im}}
\def\rec{\mathop{\rm rec}}
\def\ri{\mathop{\rm rint}}
\def\rint{\mathop{\rm rint}}
\def\sign{\mathop{\rm sign}}
\def\supp{\mathop{\rm supp}}
\def\trace{\mathop{\rm trace}}
\def\val{\mathop{\rm val}}
\def\var{\mathop{\rm var}}
\def\width{\mathop{{\rm width}}}
\def\View{\mathop{\rm View}}

\def\inf{\mathop{\rm inf}}
\def\Ker{\mathop{\rm Ker}}
\def\min{\mathop{\rm min}}
\def\max{\mathop{\rm max}}

\def\dw{\mathop{\delta w}}


\def\dPsi{\mathop{\delta \Psi}}

%%%%%%%%%%%%%%%% UNDERLINED AND OVERLINED LIMINF AND LIMSUP %%%
\def\liminfu{\mathop{\underline{\lim}}}
\def\limsupo{\mathop{\overline{\lim}}}
%%%%%%%%%%%%%%%%%%%%%%%%%%%%%%%% NUMBERS %%%%%%%%%%%%%%%%%%
\def\half{\mbox{$\frac{1}{2}$}}
\def\threeinv{\mbox{$\frac{1}{3}$}}
\def\fourinv{\mbox{$\frac{1}{4}$}}
\def\sixinv{\mbox{$\frac{1}{6}$}}
\def\1B{{\bf  1}}
\def\sbdeux#1#2{\mbox{\scriptsize$#1$}\atop\mbox{\scriptsize$#2$}}
\def\sbtrois#1#2#3{\mbox{\scriptsize$#1$}\atop\mbox{\scriptsize$#2$}\atop\mbox{\scriptsize$#3$}}
%%%%%%%%%%%%%%%%%%%%%%%%%%%%%%%%%  NUMBER SPACES  %%%%
\newcommand{\EE}{\mathbb{E}}
\newcommand{\FF}{\mathbb{F}}
\newcommand{\NN}{\mathbb{N}}
\newcommand{\RR}{\mathbb{R}}
\newcommand{\ZZ}{\mathbb{Z}}

%%%%%%%%%%%%%%%%%%%%%%%%%%%%%% OTHER MATHBB %%%%%%%%%%

\def\cC{\mathbb{C}}
\def\cN{\mathbb{N}}
\def\cP{\mathbb{P}}
\def\cQ{\mathbb{Q}}
\def\QQ{\mathbb{Q}}
\def\cR{\mathbb{R}}
\def\cZ{\mathbb{Z}}

%%%%%%%%%%%%%%%%%% OTHER TYPE OF MATHBB %%%%%%%%%%%
\newcommand{\bbC}{{\mathbb C}}
\newcommand{\bbH}{{\mathbb H}}
\newcommand{\bbI}{{\mathbb I}}
\newcommand{\bbK}{{\mathbb K}}
\newcommand{\bbN}{{\mathbb N}}
\newcommand{\bbO}{{\mathbb O}}
\newcommand{\bbP}{{\mathbb P}}
\newcommand{\bbQ}{{\mathbb Q}}
\newcommand{\bbR}{{\mathbb R}}
\newcommand{\bbS}{{\mathbb S}}
\newcommand{\bbZ}{{\mathbb Z}}


%%%%%%%%%%%%%%%%%%%%%%%%%%%% OTHER CAPS %%%%%%%%%%%
\newcommand{\gA}{\mathsf{A}}
\newcommand{\gE}{\mathsf{E}}
\newcommand{\gF}{\mathsf{F}}
\newcommand{\gG}{\mathsf{G}}
\newcommand{\gN}{\mathsf{N}}
\newcommand{\gT}{\mathsf{T}}
\newcommand{\gV}{\mathsf{V}}
\newcommand{\sG}{\mathcal{G}}
\newcommand{\sN}{\mathcal{N}}
\newcommand{\sA}{\mathcal{A}}


\newcommand{\cE}{I\!\!E}
% \newcommand{\cR}{I\!\! R} \newcommand{\cN}{I\!\!N}
%\newcommand{\cRbar}{\bar {I\!\! R}}
\newcommand{\cRbar}{\mathbb{\bar{R}}}
\newcommand{\rbar}{\overline{\mathbb{R}}}
\newcommand{\rbartosn}{\rbar^{_{\scriptstyle\sN}}}
%%%%%%%%%%%%%%%%%%%%%%%%%%%%%%%%  ENVIRONMENTS %%%%%%%%%%%%%%%%%%
\newcommand\be{\begin{equation}}
\newcommand\ee{\end{equation}}
\newcommand\ba{\begin{array}}
\newcommand\ea{\end{array}}
\newcommand{\bea}{\begin{eqnarray}}
\newcommand{\eea}{\end{eqnarray}}
\newcommand{\bean}{\begin{eqnarray*}}
\newcommand{\eean}{\end{eqnarray*}}

\newenvironment{paoplist}{\vspace{-1ex}\begin{list}{-}
{\itemsep 0mm \leftmargin 2mm \labelwidth 0mm}}
{\end{list} \vspace{-1ex}}

\newenvironment{myenumerate}{
\renewcommand{\theenumi}{\roman{enumi}}
\renewcommand{\labelenumi}{(\theenumi)}
\begin{enumerate}}{\end{enumerate}}
%%%%%%%%%%%%%%%%%%%%%%%%%%%%%%%%  SPACING %%%%%%%%%%%%%%%%%%
\newcommand{\noi}{\noindent}
\newcommand{\bs}{\bigskip}
\newcommand{\ms}{\medskip}
\newcommand{\msn}{{\bigskip \noindent}}
%%%%%%%%%%%%%%%%%%%%%%%%%%%%%%%%  MISC %%%%%%%%%%%%%%%%%%
%\newcommand{\refeq}[1]{(\ref{#1})}
%\newcommand{\eqref}[1]{(\ref{#1})}

\def\ATT{\marginpar{$\leftarrow$}}

\def\rar{\rightarrow}
\def\fmap{\rightarrow}

\def\ds{\displaystyle}
\def\disp{\displaystyle}

\def\La{\langle}
\def\la{\langle}
\def\ra{\rangle}
%%%%%%%%%%%%%%%%%%%%%%%%%% FIGURES %%%%%%%%%%%%%%%%%%%%%%%%%%%%%%%%%%

\newcommand{\mypsfig}[3]
           {\begin{figure}[hbtp]
            \centerline{\input #1}
            \caption{\rm{#2}} \label{#3}
            \end{figure}}








%%% EOF
\def\TV{\text{TV}}
%%%%%%%%%%%%%%%%%%%%%%%%%%%%%%%%%%%%%%%%%%%%%%%%%%%%%%%%%%%%
\setcounter{tocdepth}{2}
%%%%%%%%%%%%%%%%%%%%%%%%%%%%%%%%%%%%%%%%%%%%%%%%%%%%%%%%%%%%
%            TITLE

\title[]{Notes on the Benamou-Brenier formulation of Optimal Transport and geodesics in the Wasserstein space}

\author{J.M. Machado}

\date{\today}

\begin{document}

\begin{abstract}
	 
\end{abstract}
\maketitle
%\tableofcontents

\section{Derivation of the PDE by Benamou-Brenier formula}
\label{section.PDE_Benamou_Brenier}
\input{PDE_Benamou_Brenier.tex}

\appendix
\section{Technical Lemmas}
\label{appendix.technical_lemmas}
\input{technical_lemmas.tex}

%\section{Appendix}
%\label{appendix}
%\newpage
\chapter{Appendix}

\section{Auxiliary - Probability and Analysis}
This section contains definitions and results in Probability and
Analysis that are used throughout the text. These results are listed
here mostly without proofs.

\begin{definition}
	Let $d:X\times X \to \mathbb R_+$. We say that $d$ is a metric on the set $X$ if
	for all $x,y,z \in X$, the following three assertions are true:
	\begin{enumerate}[i)]
		\item $d(x,y) = 0 \iff x = y$
		\item $d(x,y) = d(y,x)$
		\item $d(x,z) \leq d(x,y) + d(y,z)$ (triangle inequality)
	\end{enumerate}
	\label{def:metric}
\end{definition}

\begin{definition}(Weak convergence)
	We say that $\mu_n \rightharpoonup \mu$ if and only if
	$\forall f$ continuous and bounded, we have
	$\int f \ d\mu_n \to \int f \ d\mu$.
	\label{def:weakconv}
\end{definition}
Note that this is equivalent to the notion of convergence in distribution,
which is more commonly known in probability.

\begin{theorem}(Portmanteau)
	\label{Portmanteau}
	Given $\mu \in \mathcal{P}(X)$, where $X$ is a metric space.
	Then, the following statements are equivalent:
	\begin{enumerate}[i)]
		\item $\mu_n \rightharpoonup \mu$;

		\item $\forall f$ bounded and uniformly continuous,
		      we have $\int f \ d\mu_n \to \int f \ d\mu$;

		\item $\forall F \subset X$ closed,
		      $\mu(F) \geq \limsup_n \mu_n(F)$;

		\item $\forall F \subset X$ open,
		      $\mu(A) \leq \liminf_n \mu_n(A)$;

		\item $\forall B$ such that $\mu(\partial B)= 0$, then
		      $\mu_n(B) \to \mu(B)$.

		      Note that every set $B$ with
		      $\mu(\partial B)=0$is called
		      a continuity set. And $\partial B$ is the boundary set of
		      B, hence $\partial B := \hat B \setminus \mathring B$.
	\end{enumerate}

\end{theorem}

\begin{theorem}
	Let $X,Y$ be metric spaces and $\mu_n \rightharpoonup \mu$.
	Given a continuous map $h:X\to Y$, then
	$h_\# \mu_n = \mu_n \circ h^{-1} \rightharpoonup h_\# \mu$.
\end{theorem}

\begin{corollary}
	If $\mu_n \rightharpoonup \mu$ with $h:X\to Y$ such that
	$\mu(D_h) = 0$ where $D_h$ is the set of points of discontinuity.
	Then, $\mu_n \circ h^{-1}\rightharpoonup \mu \circ h^{-1}$.
\end{corollary}

\begin{proposition}
	If $X$ is Polish, and $d$ is a lower semi-continuous metric on $X$. For $p \in [1,+\infty)$ and $x_0 \in X$,
	$\mu_n \rightharpoonup \mu$ and $\int_X d(x,x_0)^p d\mu_n \to \int_X d(x,x_0)^p d \mu$, if, and only if,
	$\mu_n \rightharpoonup \mu$ and $\lim_{R \to \infty} \int_{d(x,x_0)\geq R} d(x,x_0) d\mu_n \to 0$ (uniformly integrable).
\end{proposition}

\begin{definition} (Tight)
	A family of probability measures $\mathcal{A}$ is tight if for
	$\epsilon > 0$, $\exists K \subset X$ compact, such that for
	any $\mu_\alpha \in \mathcal{A}$,
	$\mu_\alpha (X \setminus K)<\epsilon$
	\label{def:tight}
\end{definition}

\begin{theorem}(Prokhorov) This theorem
	consists in two separate results.
	\label{Prokhorov}
	\begin{enumerate}[i)]
		\item If the family $\mathcal{G} =
			      \{\mu_\alpha\}_{\alpha \in \Lambda}$ is tight, then
		      $\mathcal{G}$ is sequentially pre-compact, i.e. for any
		      $(\mu_n) \subset \mathcal{G}$,
		      $\exists \mu_{n_k}\rightharpoonup \mu$, where
		      $\mu \in \overline{\mathcal{G}}$;

		\item If X is Polish and $\mathcal{G}=
			      \{\mu_\alpha\}_{\alpha \in \Lambda}\subset \mathcal{P}(X)$
		      is pre-compact. Then $\mathcal{G}$ is tight.
		      In other words, for $X$ polish, and $\mu_n \in \mathcal{P}(X)$
		      with $\mu_n \rightharpoonup \mu$, then the sequence
		      $(\mu_n)$ is tight.
	\end{enumerate}
\end{theorem}

\begin{definition}(Disintegration)

	For a Borel measurable space $X$ with a measure $\mu$.
	Given a function $f:X \to Y$. We say that the family
	$(\mu_y)_{y\in Y}$ is a Disintegration of $\mu$ according
	to $f$ if every measure $\mu_y$ is concentrated on $f^{-1}(\{y\})$, and
	for every $\phi \in C(X)$, the map $\phi \mapsto \int_X \phi d \mu_y$ is
	Borel measurable with
	\begin{equation}
		\int_X \phi \ d\mu = \int_Y \int_X \phi \ d\mu_y (x) \ d\nu(y), \quad
		\text{ where } \nu = f_\# \mu
	\end{equation}
	Note that the existence and uniqueness of disintegration families depend on the
	spaces where the probabilities are defined, to which we introduce the next theorem.
	\label{def:disintegration}
\end{definition}

\begin{theorem}(\citet{garling2018analysis} 16.10.1)
	Suppose that $X$ and $Y$ are Polish spaces, that $\mu \in \mathcal P(X)$ and that
	$f$ is a Borel measurable map from $X$ to $Y$. Then, the $f$-disintegration
	of $\mu$ exists, and is essentially unique (i.e.
	$\mu(f^{-1}(B))=0$, with
	$B := \{y \in f(X) : \mu_y \neq \mu_y'\}$ where $\mu_y$ and $\mu_y'$ are two disintegrations).
	\label{thm:disintegrationunique}
\end{theorem}

\begin{theorem}
	$f:X \to \mathbb R$ is uniformly continuous $\iff$
	$\exists \ \omega : \mathbb R_+ \to \mathbb R_+$ , such that
	$\omega$ is increasing and $\lim_{x \to 0} w(x) = 0$ with
	$|f(x) - f(y)| \leq \omega(d(x,y)), \ \forall x,y \in X$.
	We call $\omega$ the modulus of continuity.
	\label{thm:mod_continuity}
\end{theorem}

\begin{definition}(Equicontinuous)
	For a metric space $X$, the sequence of functions
	$f_n:X\to \mathbb R$ is equicontinuous if
	$\forall \epsilon >0,\ \exists \delta >0: \ d(x,y) < \delta
		\implies d(f_n(x),f_n(y))<\epsilon$ for every $n \in \mathbb N$.
	\label{def:equicontinuous}
\end{definition}

\begin{definition}(Equibounded) We say that a sequence (or family)
	of functions $(f_n)$ is equibounded,
	if $\exists M > 0 \ : \ |f_n(x)|< M < +\infty \ \forall n \in
		\mathbb N$. In words, there is a value $M$ that bounds all functions
	in the sequence.
	\label{def:equibounded}
\end{definition}

\begin{theorem}(Arzelà-Ascoli)
	If $X$ is a compact metric space with $f_n$ equicontinuous and
	equibounded, then $\exists f_{n_k}\to_{\text{unif.}}f$, where
	$f$ is continuous.
	\label{thm:arzela-ascoli}
\end{theorem}

\begin{theorem}
	Let $(X,d)$ be metric space. Thus, if $X$ is compact, then $\mathrm{Lip}(X)$ is dense in $C(X)$.
	\label{thm:lipdense}
\end{theorem}
\begin{prf} (Proof from \citet{stackoverflow1})
	Let $g:X\to \mathbb R$ be a continuous function, then
	since $X$ is compact, $g$ is uniformly continuous.
	Therefore, for any $\varepsilon>0$, one can take a
	$\delta >0$ such that $d(x,y) < \delta$ implies
	$|g(x)-g(y)|<\varepsilon$. Now, let $M = \sup_{x}|g(x)|$
	and define
	\begin{equation*}
		f(x) :=
		\sup_y g(y) - \frac{2Md(x,y)}{\delta}
	\end{equation*}
	Now, note that $f$ is Lipschitz, since
	\begin{align*}
		f(x_1) - f(x_2) &= 
		\sup_y \left(g(y) - \frac{2Md(x_1,y)}{\delta}\right) -
		\sup_y \left(g(y) - \frac{2Md(x_2,y)}{\delta} \right)\\
		&\leq
		\sup_y \frac{2M(d(x_1,y)-d(x_2,y))}{\delta})
	\end{align*}
	By the triangle inequality, $d(x_1,y) - d(x_2,y) \leq d(x_1,x_2)$, then
	\begin{equation*}
		\sup_y \frac{2M(d(x_1,y)-d(x_2,y))}{\delta} \leq 
		\sup_y \frac{2Md(x_1,x_2)}{\delta} = 
		\frac{2Md(x_1,x_2)}{\delta}
	\end{equation*}
	The same argument is valid by exchanging $x_1$ and $x_2$, so $f$ has Lipschitz constant
	$\frac{2M}{\delta}$. Next, let's prove that $\sup_x |g(x) - f(x)| < \varepsilon$.

	A first point to notice is that $f(x)\geq g(x)$, since for $y=x$, we have $f(x) = g(x)$.
	For $d(x,y) \geq \delta$,
	\begin{equation*}
		f(x) = \sup_y g(y) - \frac{2M d(x,y)}{\delta}\leq \sup_y - 2M \leq -M \leq g(x)
	\end{equation*}
	Hence $f(x)\geq g(x) \geq f(x)$, so we obtain an equality.

	For $d(x,y) < \delta$,
	\begin{equation*}
		f(x) - g(x) = \sup_y g(y) - g(x) - \frac{2M d(x,y)}{\delta}
		\leq \varepsilon -\frac{2M d(x,y)}{\delta} < \varepsilon
	\end{equation*}
	We conclude that $0< f(x) - g(x) < \varepsilon$, so $\sup_x|f(x)-g(x)|< \varepsilon$.

\end{prf}


\section{Auxiliary - Inequalities}

\begin{lemma}(Inf-Sup Inequality)
	\begin{equation}
		|\inf_{x \in A} f(x) - \inf_{x \in A} g(x)| \leq
		\sup_{x \in A}|f(x)- g(x)|
	\end{equation}
	\label{lem:infsup_ineq}
\end{lemma}
\begin{prf}
	Let's write $\sup_{x \in A}f(x)$ as $\sup_A f$ for simplicity.
	Note that $f = f - g + g$, hence,
	\begin{align*}
		\sup_A f = \sup_A f - g + g & \leq
		\sup_A (f-g) + \sup_A g \implies   \\
		\sup_A f - \sup_A g         & \leq
		\sup_A f-g \leq \sup_A |f-g|       \\
	\end{align*}
	Using the same argument for $g$, we obtain that
	\begin{equation}
		|\sup_A f - \sup_A g| \leq \sup_A |f-g|
	\end{equation}

	Finally, note that
	\begin{align*}
		|\sup_A f - \sup_A g| =
		|\inf_A (-f) - \inf_A (-g)| & =
		|-\inf_A f + \inf_A g| =                                                \\
		                            & =|\inf_A f - \inf_A g | \leq \sup_A |f-g|
	\end{align*}
\end{prf}

\begin{lemma} (Minkowski's Inequality)
	Let $X$ be a measurable space, for $p \in [1,+\infty)$ and $f,g \in L^p(X)$. Therefore,
	\begin{equation}
		||f + g||_{L^p(X)} \leq
		||f||_{L^p(X)} + 
		||g||_{L^p(X)}
	\end{equation}
	Where $||f||_{L^p(X)}^p = \int_X |f|^p d\mu$.
	\label{lem:minkowski}
\end{lemma}

%%%%%%%%%%%%%%%%%%%%%%%%%%%%%%%%%%%%%%%%%%%%%%%%%%%%%%

\bibliographystyle{plain}
\bibliography{references}


%%%%%%
% Compile bibliography with: 
%pdflatx main.tex; bibtex main; pdflatex.tex

\end{document}