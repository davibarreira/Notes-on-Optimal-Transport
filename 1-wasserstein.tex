\newpage
\section{Wasserstein Distance}

In this section we focus on how the minimal transport cost can be used as a distance metric
in the space of probability measures. Let's assume that $(X,d)$ is a Polish metric space,
$d$ is a lower semi-continuous metric on this space and $ p \in [1,+\infty)$.

\begin{definition}(Probability space with p-Moments)

  \begin{equation}
    \mathcal P_p(X) := \{
         \mu \in \mathcal P(X): \int_{X \times X} d(x,y)^p \ d \mu(x) d \mu(y) < +\infty
      \}
  \label{eq:Pp}
  \end{equation}
  Note that this is equivalent to the set of probability measures such that $\int_X d(x,x_0) \ d\mu<+\infty$
  for every $x_0 \in X$. The proof of this statement can be found
  in \citet{garling2018analysis} Proposition 21.1.1.
\end{definition}

\begin{definition}(Wasserstein Distance)

  Let $(X,d)$ be a Polish metric space, with $c:X \times X \to \mathbb R$ such that $c(x,y)=d(x,y)^p$, and
  $p \in [1,+\infty)$.
  For $\mu,\nu \in \mathcal P_p(X)$, the Wasserstein Distance is given by:
  \begin{equation}
    W_p(\mu,\nu) :=
    \left(
    \inf_{\gamma \in \Pi(\mu,\nu)}
    \int_{X \times X} d(x,y)^p \ d\gamma
    \right)^{1/p}
    \label{def:Wasserstein}
  \end{equation}
  Note that the restriction to $\mu,\nu \in \mathcal P_p(X)$ is necessary for $W_p$ to be a distance metric.
  Moreover, for $p=1$, then $c(x,y) = d(x,y)$ is a metric on $X$, therefore, for $X$ Polish, one can
  use Kantorovich-Rubinstein's Duality Theorem \ref{thm:Kantorovich-Rubinstein} to obtain:
  \begin{equation}
    W_1(\mu,\nu) =
    \sup_{\phi \in Lip_1} \int_X f d (\mu - \nu)
  \end{equation}
\end{definition}

Let's prove that $W_p$ indeed is a metric on $\mathcal P_p(X)$.

\begin{lemma} (Gluing Lemma)

  Let $(X,d)$ be a metric space. For $\mu,\nu,\rho \in \mathcal P(X)$ and
  $\gamma^+ \in \Pi(\mu,\rho)$, $\gamma^- \in \Pi(\rho,\nu)$. Then,
  $\exists \ \sigma \in \mathcal P(X \times X \times X)$ such that
  $(\pi_{x,y})_\# \sigma  = \gamma^+$.
  $(\pi_{y,z})_\# \sigma  = \gamma^-$.
  \label{lem:gluing}
\end{lemma}
\begin{prf}
  First, use disintegration (Def. \ref{def:disintegration}) with respect to $f = \pi_y$ to obtain $\gamma^+_y$ and $\gamma^-_y$.
  We know that such
  disintegration exists and is essentially unique since $X$ is Polish (see Theorem \ref{thm:disintegrationunique}).
  Note that disintegrated measures are actually
  defined on $X \times \{y\} \subset X \times X$, but, by abuse of notation, we'll consider that they
  are measures on $X$, and $y$ is only an index.

  Therefore, make $\sigma = \gamma_y^+ \otimes \rho \otimes \gamma_y^-$, and let $\phi:X \times X \to \mathbb R$
  be a measurable function.
  Hence:
  \begin{align*}
    \int_{X \times X \times X} \phi(x,y) \ d\sigma & \overset{\text{Fubini}}{=}
    \int_X \int_X \int_X \phi(x,y) \
    d\gamma_y^+(x) \otimes \rho(y) \otimes \gamma_y^-(z)                                        \\
                                                                  & \overset{\text{Indep.}}{=}
    \int_X d\gamma_y^-(z) \ \int_X \int_X \phi(x,y) \
    d\gamma_y^+(x) \otimes \rho(y)                                                              \\
                                                                  & \overset{\text{Disint.}}{=}
    \int_X d\gamma_y^-(z) \ \int_{X \times X} \phi(x,y) \
    d\gamma^+(x,y)                                                                              \\
                                                                  & \overset{\text{\hfill}}{=}
    \int_{X\times X} \phi(x,y) \
    d\gamma^+(x,y)
  \end{align*}
  Since $\phi(x,y)$ is arbitrary, then by Corollary \ref{cor_marginals}, we can conclude that
  $(\pi_{x,y})_\# \sigma  = \gamma^+$. By the same argument, we obtain
  $(\pi_{y,z})_\# \sigma  = \gamma^-$, which concludes our proof.

\end{prf}

\begin{proposition}
  $W_p(\cdot,\cdot)$ is a metric on $\mathcal P_p(X)$.
\end{proposition}
\begin{prf} Let's prove each of the three properties that categorize a metric.

  \vspace{5mm}
  \noindent	i) $d(x,y) = 0 \iff x = y$.

  \vspace{5mm}
  \noindent
  If $\mu=\nu$, then
  $(id,id)_\# \mu = \gamma$, hence $\int_{X \times X} d(x,y)^p
    \ d\gamma = \int_{X \times X} d(x,x)^p \ d\mu =0$.

  \vspace{5mm}
  If $W_p(\mu,\nu)=0$, then $\int_{X\times X}d(x,y)^p \ d\gamma = 0$. Therefore, $\gamma$ is concentrated
  on $\{x=y\}$, otherwise, there would exist a set $\ A \times B$ such that $\gamma(A \times B)>0$ and $x\neq y$.
  Therefore $\int_X d(x,y)^p d\gamma >0$.

  Since $\gamma$ is concentrated on $\{x=y\}$, then for any set Borel set $K \subset X$:
  \begin{equation*}
    \gamma(K) = \int_{X\times X} \mathbbm 1_K(x,y) \ d\gamma =
    \int_{x=y} \mathbbm 1_K(x,y) \ d\gamma = \int_{x=y} \mathbbm 1_K(x) \ d\mu
    = \int_{x=y} \mathbbm 1_K (y) \ d\nu
  \end{equation*}
  We can conclude that $\mu(K)=\nu(K)$ for every Borel set $K$, therefore $\mu=\nu$ almost everywhere.

  \vspace{5mm}

  \noindent	ii) $d(x,y)=d(y,x)$.
  \begin{equation*}
    W_p(\mu,\nu) = \left(\int_{X \times X} d(x,y)^p d\gamma \right)^{1/p} =
    \left(\int_{X \times X} d(y,x)^p d\gamma\right)^{1/p} = W_p(\nu,\mu)
  \end{equation*}

  \vspace{5mm}
  \noindent	iii) $d(x,z) \leq d(x,y) + d(y,z)$.

  Let $\mu,\nu,\rho \in \mathcal P_p(X)$, and $\gamma^+ \in \Pi(\mu,\rho)$, $\gamma^- \in \Pi(\rho,\nu)$ are
  the optimal transport plans for the respective measures.
  Using the Gluing Lemma \ref{lem:gluing}, we know that there exists a measure
  $\sigma \in \mathcal P(X \times X \times X)$, where
  $(\pi_{x,y})_\# \sigma = \gamma^+$ and
  $(\pi_{y,z})_\# \sigma = \gamma^-$. Also, let $\gamma := (\pi_{x,z})_\# \sigma$. Hence,
  \begin{align*}
    W_p(\mu,\nu) & \quad \leq
    \left(
      \int_{X \times X} d(x,z)^p \ d\gamma
    \right)^{1/p} =
    \left(
      \int_{X \times X} d(x,z)^p \ d(\pi_{x,z})_\# \sigma
    \right)^{1/p}\\
    & \underset{Thm. \ref{thm:pushforward}}{=}
    \left(
    \int_{X \times X \times X} d(x,z)^p \ d \sigma
    \right)^{1/p}\\
    & \quad \leq
    \int_{X^3} (d(x,y)+d(y,z))^p \ d \sigma \\
    &=
    ||
    d \circ (\pi_{x,y})(x,y,z) -
    d \circ (\pi_{y,z})(x,y,z)
    ||_{L^p(\sigma)} \\
    & \underset{\ref{lem:minkowski}}{\leq}
    ||
    d \circ (\pi_{x,y})(x,y,z)
    ||_{L^p(\sigma)} +
    ||
    d \circ (\pi_{y,z})(x,y,z)
    ||_{L^p(\sigma)} \\
    & =
    \left(
      \int_{X^3} d(x,y)^p \ d\sigma
    \right)^{1/p} +
    \left(
      \int_{X^3} d(y,z)^p \ d\sigma
    \right)^{1/p} \\
    & =
    \left(
      \int_{X^2} d(x,y)^p \ d\gamma^+
    \right)^{1/p} +
    \left(
      \int_{X^2} d(y,z)^p \ d\gamma^-
    \right)^{1/p} \\
    & =
    W_p(\mu,\rho) + W_p(\rho,\nu)
  \end{align*}
  Which proves the triangle inequality for the Wasserstein distance.

\end{prf}

\begin{definition} (Wasserstein Space)
  For a Polish space $X$, we call $\mathcal P_p(X)$ a Wasserstein space if it is endowed with
  the p-Wasserstein metric. Note that is also common to see this space symbolized by $\mathcal W_p(X)$.
\end{definition}

\begin{proposition}
  For a bounded Polish space $X$, $p \in [1,+\infty)$, $\mu,\nu \in \mathcal P_p(X)$ and $C\in \mathbb R_+$, then
  \begin{equation}
    W_1(\mu,\nu) \leq W_p(\mu,\nu) \leq CW_1(\mu,\nu)^{1/p}
  \end{equation}
  \label{prop:ineqwasserstein}
\end{proposition}
\begin{prf}
  Let $p\leq q \in [1,+\infty)$ and $\gamma \in \Pi(\mu,\nu)$. Hence, $\phi(x)=x^{q/p}$ is a convex function, so by Jensen's
  inequality:
  \begin{align*}
    \phi\left(
    \int d(x,y)^p d\gamma
    \right)^{1/q} =
    \left(
    \int d(x,y)^p d\gamma
    \right)^{1/p}
     & \leq
    \left(
    \int \phi(d(x,y)^p) d\gamma
    \right)^{1/q} \\
     & =
    \left(
    \int (d(x,y)^q) d\gamma
    \right)^{1/q}
  \end{align*}
  This implies that $W_p(\mu,\nu) \leq W_q(\mu,\nu)$, when $p\leq q$. In particular,
  $W_1(\mu,\nu)\leq W_p(\mu,\nu)$ for $p\geq 1$.

  Now, since $X$ is bounded, then $d(x,y) \leq \sup_{x,y \in X}d(x,y) = d(X)$. Hence,
  \begin{gather*}
    d(x,y)^p \leq d(X)^{p-1}d(x,y) \\
    \therefore
    \\
    \left(
    \int d(x,y)^p d\gamma
    \right)^{1/p} \leq
    \left(
    \int d(x,y) d\gamma
    \right)^{1/p}d(X)^{\frac{p-1}{p}}
  \end{gather*}
  Therefore, we conclude that $W_p(\mu,\nu)\leq d(X)^{\frac{p-1}{p}} W_1(\mu,\nu)^{1/p}$

\end{prf}

Next, let's present some of the topological properties of such space.	A first thing to note is that
for probability spaces, the notion of weak convergence can be made more strict with the following lemma:

\begin{lemma}
  For a space of probability measures, we say that $\mu_n$ converges weakly to $\mu$, i.e.
  $\mu_n \rightharpoonup \mu \iff \ \forall f \in C_c(X), \ \int f \ d\mu_n \to \int f \ d\mu$, where
  $C_c(X)$ is the space of continuous functions with compact support. Note that
  $C_c(X) \subset C_0(X) \subset C_b(X)$.
  \label{lem:weakconvergenceCc}
\end{lemma}
\begin{prf}

  $\implies)$ If $\mu_n \rightharpoonup \mu$	, then $f \in C_c(X)\subset C_b(X)$, hence $\int f d\mu_n \to \int f d\mu$.

  \vspace{5mm}
  $\impliedby)$ Suppose that $\forall f \in C_c(X),\ \int f d\mu_n \to \int f d\mu$. Hence, note that for
  any constant $M$, $\int f + M d\mu_n = \int f d\mu_n + C \to \int f d\mu + C$.
  Take $g \in C_b(X)$ and make $g' = g + C \geq 0$ and
  $g' \mathbbm 1_{[-k,k]} = f_k \in  C_c(X)$. Which implies that $f_k \uparrow g'$.
  Now,
  \begin{align*}
    \left|\int g d\mu_n - \int g d\mu \right| & =
    \left|\int g' d\mu_n - \int g' d\mu \right|      \\
                                              & \leq
    \left|\int g' d\mu_n - \int f_k d\mu_n \right| +
    \left|\int f_k d\mu_n - \int f_k d\mu \right| +
    \left|\int f_k d\mu - \int g' d\mu \right|
  \end{align*}
  Since $f_k \in C_c(X)$, then for $n$ big enough,
  $\left|\int f_k d\mu - \int f_k d\mu_n \right|< \epsilon$. Therefore,
  \begin{align*}
    \left|\int g d\mu_n - \int g d\mu \right| \leq
    \left|\int g' d\mu_n - \int f_k d\mu_n \right| +
    \epsilon +
    \left|\int f_k d\mu - \int g' d\mu \right|
  \end{align*}
  Since $f_k \uparrow g'$, then,
  by the Monotone Convergence Theorem,

  \begin{gather*}
    \lim_{k\to +\infty}
    \left|\int g' d\mu_n - \int f_k d\mu_n \right| = 0 \\
    \lim_{k\to +\infty}
    \left|\int f_k d\mu - \int g' d\mu \right| = 0 \\
    \therefore
  \end{gather*}

  \begin{equation*}
    \lim_{k\to +\infty}\left|\int g d\mu_n - \int g d\mu \right| =
    \left|\int g d\mu_n - \int g d\mu \right| \leq
    \epsilon
  \end{equation*}
\end{prf}

\begin{theorem}
  Let $(X,d)$ be a Polish compact space with $\mu_n,\mu \in P_p(X)$ and
  $p \in [1,+\infty)$, then $W_p(\mu_n,\mu)\to 0 \iff \mu_n \rightharpoonup \mu$.
  \label{thm:compactwassersteinconv}
\end{theorem}
\begin{prf}

  $\implies)$ Let $W_p(\mu_n,\mu)\to 0$. Since $X$ is compact and $c$ is a continuous function,
  by Theorem \ref{thm:Santambrogio1.4} the Kantorovich Problem has a solution. Also, by Theorem \ref{thm:compactstrongduality},
  we obtain that $\max(\mathrm{DP})=\min(\mathrm{KP})$. First, we prove for $p=1$.
  In this case, using the Lipschitz version of DP:
  \begin{equation*}
    W_1(\mu,\nu)=
    \max \left \{
    \int_X \phi \ d\mu - \int_X \phi \ d\nu \ :
    \phi \in Lip_1(X)
    \right \} \to 0
  \end{equation*}
  This implies that for any $f \in \text{Lip}_1, \int f d\mu_n \to \int f d\mu$. Note that, by linearity,
  the same is true for any Lipschitz function. Since $X$ is compact, then Lipschitz functions are
  dense on $C(X)$ (see Theorem \ref{thm:lipdense}), which leads us to conclude that $\mu_n \rightharpoonup \mu$
  (by Portmanteau \ref{Portmanteau}). Now, by Proposition \ref{prop:ineqwasserstein},
  the same is valid for any $p\geq 1$.

  $\impliedby)$ Let $\mu_n \rightharpoonup \mu$. Define a subsequence $\mu_{n_k}$ such that
  $\lim_k W_1(\mu_{n_k},\mu)=\limsup_n W_1(\mu_n,\mu)$. By the same arguments already used,
  we know that for each $\mu_{n_k}$ there is a $\phi_{n_k} \in \text{Lip}_1$ such that
  $W_1(\mu_{n_k},\mu) = \int_X \phi_{n_k}d(\mu_{n_k}-\mu)$.

  For an arbitrary $\epsilon >0$, make $\delta = \epsilon$. Since $\phi_{n_k}$ is 1-Lipschitz,
  if $d(x,y) < \delta$, then
  $|\phi_{n_k}(x) - \phi_{n_k}(y)| \leq d(x,y) < \epsilon, \ \forall k \in \mathbb N$. Therefore, the sequence is
  Equicontinuous.

  Also, for $x_0 \in X$, we can make $\phi_{n_k}'(x):= \phi_{n_k}(x) - \phi_{n_k}(x_0)$. Note that these functions are
  1-Lipschitz and still satisfy
  $W_1(\mu_{n_k},\mu) = \int_X \phi_{n_k}'d(\mu_{n_k}-\mu)$. Hence, let's use $\phi_{n_k}'$ as our new subsequence.
  In this case,
  \begin{equation*}
    |\phi_{n_k}'(x)| =
    |\phi_{n_k}(x) - \phi_{n_k}(x_o)|
    \leq d(x,x_o) \leq d(X) <+\infty
  \end{equation*}
  This implies that this sequence of $\phi_{n_k}'$ is Equibounded.
  With this, we can use Arzelà-Ascoli Theorem (\ref{thm:arzela-ascoli})
  to obtain a sub-subsequence that converges uniformly to a $\phi \in \text{Lip}_1(X)$.
  Replace and relabel the original subsequence, obtaining:
  \begin{align*}
     & W_1(\mu_{n_k},\mu) = \int_X \phi_{n_k}d(\mu_{n_k}-\mu) \\
     & =
    \left|
    \int_X \phi_{n_k}d\mu_{n_k}+
    \int_X \phi d\mu_{n_k} -
    \int_X \phi d\mu_{n_k} +
    \int_X \phi d\mu -
    \int_X \phi d\mu -
    \int_X \phi_{n_k}d\mu
    \right|                                                        \\
     & \leq
    \underbrace{
      \left|
      \int_X \phi_{n_k}d\mu_{n_k} -
      \int_X \phi d\mu_{n_k}
      \right|}
    _{\text{Goes to }0, \text{due to } \phi_{n_k}\to_u \phi}+
    \underbrace{
      \left|
      \int_X \phi d\mu-
      \int_X \phi_{n_k} d\mu
      \right|}
    _{\text{Goes to }0, \text{due to } \phi_{n_k}\to_u \phi}+
    \underbrace{
      \left|
      \int_X \phi d\mu_{n_k} -
      \int_X \phi d\mu
      \right|}
    _{\text{Goes to }0, \text{due to } \mu_{n_k}\rightharpoonup \mu}
  \end{align*}

  Therefore $\limsup_n W_1(\mu_n,\mu) \leq 0 \implies W_1(\mu_n\mu) \to 0$. To conclude the proof
  for any $p \in [1,+\infty)$, we use Proposition \ref{prop:ineqwasserstein}:
  \begin{equation*}
    0 \leq W_p(\mu_n,\mu) \leq CW_1(\mu_n,\mu)^{1/p} \leq 0
  \end{equation*}
\end{prf}

\begin{theorem}
  For $X \subset \mathbb R^n$, $\mu_n,\mu \in \mathcal P_p(X)$, $x_0 \in X$ and
  $d$ is metric on $X$, then
  \begin{equation}
    W_p(\mu_n,\mu) \to 0 \iff \int_X d(x,x_0)^p d\mu_n \to \int_X d(x,x_0)^p d\mu
    \text{ and } \mu_n \rightharpoonup \mu
  \end{equation}
  \label{thm:convwasserstein}
\end{theorem}

\begin{prf}

  $\implies)$ Let $W_p(\mu_n,\mu)\to 0$. Since $X$ is Polish, and $c$ is a continuous function,
  by Theorem \ref{thm:existanceKPpolish} the Kantorovich Problem has a solution. Also, by Theorem \ref{thm:strongerDuality},
  we obtain that $\sup(\mathrm{DP})=\min(\mathrm{KP})$. We know that
  $W_p(\mu_n,\mu) \geq W_1(\mu_n,\nu)\geq 0$, hence, using the Lipschitz version of the Dual Problem for $W_1$:
  \begin{equation*}
    \sup \left \{
    \int_X \phi \ d\mu_n - \int_X \phi \ d\mu \ :
    \phi \in Lip_1(X)
    \right \} \to 0
  \end{equation*}

This implies that for any $f \in \text{Lip}_1, \int f d\mu_n \to \int f d\mu$. Note that, by linearity,
the same is true for any Lipschitz function, not only $\text{Lip}_1$. Finally, since Lipschitz functions are
dense on $C_c(X)$ (see Theorem \ref{thm:lipdense}),
we can use Lemma \ref{lem:weakconvergenceCc} to conclude that $\mu_n \rightharpoonup \mu$.

To prove the other condition (i.e.
$\int_X d(x,x_0)^p d\mu_n \to \int_X d(x,x_0)^p d\mu$),
define $\delta_{x_0}$ as a measure with mass on $x_0$. Which means that the optimal transport plan
$\gamma_n$ is in $\Pi(\mu_n,\delta_{x_0})$. This implies that $\gamma_n(x,y) = 0$ for any $y \neq x_0$. Therefore
\begin{align*}
  W_p(\mu_n,\delta_{x_0})^p = \int_{X \times X} d(x,y)^p d\gamma_n & = \int_{X \times \{x_0\}} d(x,y)^p d\gamma_n                              \\
  & = \int_X d(x,x_0)^p d\mu_n \to W_p(\mu,\delta_{x_0})^p = \int_X d(x,x_0)^p d\mu
\end{align*}
Where we used the fact that $W(\mu_n, \delta_{x_0}) \to W(\mu,\delta_{x_0})$, which is true since
$W(\mu_n,\delta_{x_0}) - W(\mu,\delta_{x_0}) \leq W(\mu_n,\mu)$.

\vspace{5mm}
$\impliedby)$ Consider now that $\mu_n \rightharpoonup \mu$ and
Define $\pi_R :X \to \overline{\text{B}(R)}$, which is the projection on the closed ball with radius $R$
centered at $x_0$. Since $W_p(\cdot,\cdot)$ is a metric, we have:
\begin{gather*}
  W_p(\mu_n,\mu) \leq
  W_p(\mu_n,(\pi_R)_\#\mu_n) +
  W_p((\pi_R)_\#\mu_n,(\pi_R)_\#\mu)+
  W_p((\pi_R)_\#\mu_n,\mu)
\end{gather*}

For sake of clarity in the proof, let's define, without loss of generalization, that $d(x,x_0) = |x|$ and
$d(x,y) = |x-y|$.
Now, note that $|x - \pi_R(x)| =|x| - |x| \wedge R$ and
that the plan $(id,\pi_R)_\# \mu$ is a possible solution to the OT Problem of transporting
$\mu$ to $(\pi_R)_\#\mu$. Therefore:
\begin{align*}
  W_p(\mu,(\pi_R)_\# \mu)^p & \leq
  \int_{X \times X} |x-y|^p d (id,\pi_R)_\# \mu =
  \int_{(id,\pi_R)^{-1}(X\times X)} |x-\pi_R(x)|^p d\mu \\
                            & =
  \int_{X} |x-(x\wedge R)|^p d\mu =
  \int_{B(R)^c} (|x|-R)^p d\mu                                    \\
\end{align*}
And using the same arguments:
\begin{align*}
  W_p(\mu_n,(\pi_R)_\# \mu_n)^p & \leq
  \int_{B(R)^c} (|x|-R)^p d\mu_n
\end{align*}
Now, note that
\begin{equation*}
  \int_X |x|^p - (|x|\wedge R)^p d\mu = \int_{B(R)}|x|^p -|x|^p d\mu
  + \int_{B(R)^c} |x|^p - R^p d\mu \leq \int_{B(R)^c} |x|^p d\mu
\end{equation*}
Since $\mu_n, \mu \in \mathcal P_p(X)$, we know that
$\int_{X}|x|^p d\mu = C < +\infty$ and
$\int_{X}|x|^p d\mu_n = C < +\infty$ then
\begin{align*}
  \int_{B(R)^c} |x|^p d\mu = C - \int_{B(R)}|x|^p d\mu \quad \therefore \quad
  \lim_{R \to 0}
  \int_{B(R)^c}|x|^p = 0
\end{align*}
Using that $(|x|- R)^p \leq |x|^p - (|x|\wedge R)^p$ for every $x \in B(R)^c$, we get
\begin{align*}
  W_p(\mu_n,(\pi_R)_\# \mu)^p \leq
  \int_{B(R)^c} (|x|-R)^p d\mu_n \leq
  \int_{B(R)^c} |x|^p-R^p d\mu_n \leq
  \int_{B(R)^c}|x|^p
\end{align*}
Now, note that since $\int|x|^p \mu_n \to \int|x|^p d\mu$ and that $(|x|\wedge R)$ is continuous and bounded,
\begin{align*}
  \lim_n
  W_p(\mu_n,(\pi_R)_\#\mu_n) & \leq
  \lim_n
  \int_{B(R)^c} (|x|-R)^p d\mu_n    \\
                             & \leq
  \lim_n
  \int_{B(R)^c} |x|^p-R^p d\mu_n =
  \int_{B(R)^c} |x|^p-R^p d\mu \leq
  \int_{B(R)^c} |x|^p d\mu
\end{align*}
Hence,
\begin{align*}
  \lim_R \lim_n (W_p(\mu_n,(\pi_R)_\#\mu_n) \leq \lim_R
  \int_{B(R)^c} |x|^p d\mu = 0 \\
  \lim_R (W_p(\mu,(\pi_R)_\#\mu) \leq \lim_R
  \int_{B(R)^c} |x|^p d\mu = 0
\end{align*}

Lastly, note that since $\overline{B(R)}$ is compact, then we can use Theorem \ref{thm:compactwassersteinconv}
to stablish that
\begin{equation*}
  \lim_n W_p((\pi_R)_\#\mu_n,(\pi_R)_\#\mu) = 0
\end{equation*}

We can then conclude that
\begin{align*}
  \limsup_n W_p(\mu_n,\mu) &\leq
  \lim_R \limsup_n (
  W_p(\mu_n,(\pi_R)_\#\mu_n) \\ & \quad \quad+
  W_p((\pi_R)_\#\mu_n,(\pi_R)_\#\mu)\\ & \quad \quad+
  W_p((\pi_R)_\#\mu_n,\mu)) \\
  &= 0
\end{align*}

\end{prf}

The Theorem above was proved for $X \subset \mathbb R^d$,
but a more general result can be proven for Polish spaces. Such result is presented below without a proof.
The proof can be found in \citet{villani2008optimal} under Theorem 6.9.

\begin{theorem}

  For $(X,d)$ a Polish metric space, $\mu_n,\mu \in \mathcal P_p(X)$ and $x_0 \in X$. Then
  \begin{equation}
    W_p(\mu_n,\mu) \to 0 \iff \int_X d(x,x_0)^p d\mu_n \to \int_X d(x,x_0)^p d\mu
    \text{ and } \mu_n \rightharpoonup \mu
  \end{equation}
  \label{thm:polishwmetrize}
\end{theorem}

Let's just put some words on these last two theorems we introduced.
We showed that the p-Wasserstein distance metrizes weak convergence
of probability measures in the space $\mathcal P_p(X)$, with $(X,d)$ a Polish space.
Such property is very useful and is not present in many other commonly used distances such as
Total Variation and the Kullback-Leibler Divergence.

Yet, there are many other ways to metrize weak convergence, such as Prokhorov's distance and bounded
Lipschitz distance. So, besides this \textit{metrization}, \citet{villani2008optimal}
gives the following reasons that make $W_p$ such an interesting metric:
\begin{enumerate}[(i)]
  \item It's definition makes it a natural choice in OT problems;
  \item The distance has a rich duality, especially for $p=1$;
  \item Since it's defined with an infimum, it is easy to bound from above;
  \item Wasserstein distances incorporate information of the ground geometry.
\end{enumerate}

For applications in Data Science, the equivalence with weak convergence and the
incorporation of the ground geometry are probably the most attractive characteristics.
Figure \ref{fig:wl-kl}
highlights how $W_p$ takes into account the underlying geometry compared
to the Kullback-Leibler divergence, which does not.

\citet{villani2008optimal} also points out that:
\begin{quote}
  As a geenral rule, the $W_1$ distance is more flexible and easier to bound,
  while the $W_2$ distance better reflects geometric features (at least for problems
  with a Riemannian flavor), and is better adapted when there is more structure; it also
  scales better with the dimension. Results in $W_2$ distance are usually stronger, and more
  difficult to establish, than results in $W_1$ distance.
\end{quote}


\begin{figure}[H]
  \centering
  \def\svgscale{0.7}
  \includesvg[inkscapelatex=false]{Figures/wassersteingeometry.svg}
	\caption{Comparison between Wasserstein distance and KL Divergence, based on \citet{montavon2016boltzmann}.
  On the left,
  there is a large overlap between the two distributions, but a large geometrical distance for a portion. On the right,
  there is much less overlap, but the whole distribution is geometrically closer. These two
  cases clearly highlight how $W_p$ incorporates geometrical information while $KL$ doesn't.}
	\label{fig:wl-kl}
\end{figure}

Before finishing our initial exposition on the Wasserstein distance, let's prove some more relevant results.

\begin{corollary}(Continuity of $W_p$)
  Let $(X,d)$ be a Polish metric space, and $p \in [1,+\infty)$, then $W_p$ is continuous on $P_p(X)$, i.e.
  if $\mu_k \rightharpoonup \mu$, $\nu_k \rightharpoonup \nu$, and
  $\int d(x_0, x)^p d\mu_k(x) \to \int d(x_0,x)^p d\mu$,
  $\int d(x_0, x)^p d\nu_k(x) \to \int d(x_0,x)^p d\nu$
  , then
  \begin{equation}
    W_p(\mu_k,\nu_k) \to W_p(\mu,\nu)
  \end{equation}
\end{corollary}
\begin{prf}
  Just note that
  \begin{equation*}
    W_p(\mu_k,\nu_k) \leq W_p(\mu_k,\mu) + W_p(\mu,\nu) + W_p(\nu_k,\nu)
  \end{equation*}
  Hence, taking the limit and using Theorem \ref{thm:polishwmetrize}
  \begin{equation*}
    \lim_{k\to +\infty} 
    W_p(\mu_k,\nu_k) \leq  W_p(\mu,\nu)
  \end{equation*}
  We can perform the same steps, but now for the reverse inequality
  \begin{equation*}
    W_p(\mu,\nu) \leq W_p(\mu,\mu_k) + W_p(\mu_k,\nu_k) + W_p(\nu_k,\nu)
  \end{equation*}
  And again, taking the limit, we conclude that
  \begin{equation*}
    W_p(\mu_k,\nu_k) \to W_p(\mu,\nu).
  \end{equation*}

\end{prf}

Note that if we didn't have
  $\int d(x_0, x)^p d\mu_k(x) \to \int d(x_0,x)^p d\mu$ and
  $\int d(x_0, x)^p d\nu_k(x) \to \int d(x_0,x)^p d\nu$
in the hypothesis, we would instead conclude that
\begin{equation*}
  W_p(\mu,\nu)\leq \liminf W_p(\mu_k,\nu_k).
\end{equation*}

The last results that we want to prove is that for a Polish space $X$, then the
p-Wasserstein space is also complete and separable. To prove this, we first
need the following non-trivial lemma.

\begin{lemma}(\citet{villani2008optimal} 6.14 - Cauchy sequences in $W_p$ are tight)

  Let $X$ be a Polish space, $p \geq 1$ and $(\mu_n)_{n\in \mathbb N}$ a Cauchy sequence
  in $(P_p(X), W_p)$. Then $(\mu_n)$ is tight.
\end{lemma}

% \begin{prf}
%   First, note that since $W_p \geq W_1$, then $(\mu_n)$ is also a Cauchy sequence in $W_1$.
%   Hence, for $\varepsilon >0$, there exists $N \in \mathbb N$ such that for $k \geq N$, then
%   \begin{equation*}
%     W_1(\mu_N,\mu_k) < \varepsilon^2.
%   \end{equation*}
%   Now, take the subset $\{\mu_1,...,\mu_N\}$. It's clear that it is tight, hence, there exists
%   a compact set $K$ such that $\mu_j(X \setminus K) < \varepsilon$ for all $j \in \{1,...,N\}$.
%   Since $K$ is compact, it can be covered by a finite number of balls, thus
%   \begin{equation*}
%     K \subset B(x_1,\varepsilon)\cup ... \cup B(x_m,\varepsilon) := U.
%   \end{equation*}
%   Next, cover $U$ with $U_\varepsilon := \{x \in X: d(x,U) < \varepsilon\}$, and define
%   \begin{equation*}
%     \phi(x):= \max \left \{0 \ ;\ 1- \frac{d(x,U)}{\varepsilon} \right \}.
%   \end{equation*}
%   This implies that $\mathbbm 1_U \leq \phi \leq \mathbbm 1_{U_\varepsilon}$ and
%   $\frac{\phi(x)}{\varepsilon} \in \text{Lip}_1(X)$. Hence, we can use the Kantorovich-Rubinstein
%   and obtain the following inequality:
%   \begin{align*}
%     \mu_k (U_\varepsilon) &\geq
%     \int \phi d\mu_k
%     \\ &= 
%     \int \phi d\mu_j + \left(
%       \int \phi d\mu_k - \int \phi d \mu_j
%     \right)
%     \\ &\geq 
%     \mu_j(U) -
%     \frac{W_1(\mu_k,\mu_j)}{\varepsilon}.
%   \end{align*}
%   If $k\geq N$, then for $j = N$ we have 
%   $W_1(\mu_k,\mu_j) < \varepsilon^2$, and if $k \leq N$, then 
% \end{prf}

Finally, we can now prove the following theorem.

\begin{theorem}(\citet{villani2008optimal} 6.18)

  Let $X$ be a \textbf{complete} and \textbf{separable} metric space and $p \in [1,+\infty)$. Then $(P_p(X),W_p)$
  is also \textbf{complete} and \textbf{separable}. Moreover, any probability measure on
  $P_p(X)$ can be approximated by a sequence of probability measures with finite support.
\end{theorem}