\newpage
\chapter{Appendix}

\section{Auxiliary - Probability and Analysis}
This section contains definitions and results in Probability and
Analysis that are used throughout the text. These results are listed
here mostly without proofs.

\begin{definition}
	Let $d:X\times X \to \mathbb R_+$. We say that $d$ is a metric on the set $X$ if
	for all $x,y,z \in X$, the following three assertions are true:
	\begin{enumerate}[i)]
		\item $d(x,y) = 0 \iff x = y$
		\item $d(x,y) = d(y,x)$
		\item $d(x,z) \leq d(x,y) + d(y,z)$ (triangle inequality)
	\end{enumerate}
	\label{def:metric}
\end{definition}

\begin{definition}(Weak convergence)
	We say that $\mu_n \rightharpoonup \mu$ if and only if
	$\forall f$ continuous and bounded, we have
	$\int f \ d\mu_n \to \int f \ d\mu$.
	\label{def:weakconv}
\end{definition}
Note that this is equivalent to the notion of convergence in distribution,
which is more commonly known in probability.

\begin{theorem}(Portmanteau)
	\label{Portmanteau}
	Given $\mu \in \mathcal{P}(X)$, where $X$ is a metric space.
	Then, the following statements are equivalent:
	\begin{enumerate}[i)]
		\item $\mu_n \rightharpoonup \mu$;

		\item $\forall f$ bounded and uniformly continuous,
		      we have $\int f \ d\mu_n \to \int f \ d\mu$;

		\item $\forall F \subset X$ closed,
		      $\mu(F) \geq \limsup_n \mu_n(F)$;

		\item $\forall F \subset X$ open,
		      $\mu(A) \leq \liminf_n \mu_n(A)$;

		\item $\forall B$ such that $\mu(\partial B)= 0$, then
		      $\mu_n(B) \to \mu(B)$.

		      Note that every set $B$ with
		      $\mu(\partial B)=0$is called
		      a continuity set. And $\partial B$ is the boundary set of
		      B, hence $\partial B := \hat B \setminus \mathring B$.
	\end{enumerate}

\end{theorem}

\begin{theorem}
	Let $X,Y$ be metric spaces and $\mu_n \rightharpoonup \mu$.
	Given a continuous map $h:X\to Y$, then
	$h_\# \mu_n = \mu_n \circ h^{-1} \rightharpoonup h_\# \mu$.
\end{theorem}

\begin{corollary}
	If $\mu_n \rightharpoonup \mu$ with $h:X\to Y$ such that
	$\mu(D_h) = 0$ where $D_h$ is the set of points of discontinuity.
	Then, $\mu_n \circ h^{-1}\rightharpoonup \mu \circ h^{-1}$.
\end{corollary}

\begin{proposition}
	If $X$ is Polish, and $d$ is a lower semi-continuous metric on $X$. For $p \in [1,+\infty)$ and $x_0 \in X$,
	$\mu_n \rightharpoonup \mu$ and $\int_X d(x,x_0)^p d\mu_n \to \int_X d(x,x_0)^p d \mu$, if, and only if,
	$\mu_n \rightharpoonup \mu$ and $\lim_{R \to \infty} \int_{d(x,x_0)\geq R} d(x,x_0) d\mu_n \to 0$ (uniformly integrable).
\end{proposition}

\begin{definition} (Tight)
	A family of probability measures $\mathcal{A}$ is tight if for
	$\epsilon > 0$, $\exists K \subset X$ compact, such that for
	any $\mu_\alpha \in \mathcal{A}$,
	$\mu_\alpha (X \setminus K)<\epsilon$
	\label{def:tight}
\end{definition}

\begin{theorem}(Prokhorov) This theorem
	consists in two separate results.
	\label{Prokhorov}
	\begin{enumerate}[i)]
		\item If the family $\mathcal{G} =
			      \{\mu_\alpha\}_{\alpha \in \Lambda}$ is tight, then
		      $\mathcal{G}$ is sequentially pre-compact, i.e. for any
		      $(\mu_n) \subset \mathcal{G}$,
		      $\exists \mu_{n_k}\rightharpoonup \mu$, where
		      $\mu \in \overline{\mathcal{G}}$;

		\item If X is Polish and $\mathcal{G}=
			      \{\mu_\alpha\}_{\alpha \in \Lambda}\subset \mathcal{P}(X)$
		      is pre-compact. Then $\mathcal{G}$ is tight.
		      In other words, for $X$ polish, and $\mu_n \in \mathcal{P}(X)$
		      with $\mu_n \rightharpoonup \mu$, then the sequence
		      $(\mu_n)$ is tight.
	\end{enumerate}
\end{theorem}

\begin{definition}(Disintegration)

	For a Borel measurable space $X$ with a measure $\mu$.
	Given a function $f:X \to Y$. We say that the family
	$(\mu_y)_{y\in Y}$ is a Disintegration of $\mu$ according
	to $f$ if every measure $\mu_y$ is concentrated on $f^{-1}(\{y\})$, and
	for every $\phi \in C(X)$, the map $\phi \mapsto \int_X \phi d \mu_y$ is
	Borel measurable with
	\begin{equation}
		\int_X \phi \ d\mu = \int_Y \int_X \phi \ d\mu_y (x) \ d\nu(y), \quad
		\text{ where } \nu = f_\# \mu
	\end{equation}
	Note that the existence and uniqueness of disintegration families depend on the
	spaces where the probabilities are defined, to which we introduce the next theorem.
	\label{def:disintegration}
\end{definition}

\begin{theorem}(\citet{garling2018analysis} 16.10.1)
	Suppose that $X$ and $Y$ are Polish spaces, that $\mu \in \mathcal P(X)$ and that
	$f$ is a Borel measurable map from $X$ to $Y$. Then, the $f$-disintegration
	of $\mu$ exists, and is essentially unique (i.e.
	$\mu(f^{-1}(B))=0$, with
	$B := \{y \in f(X) : \mu_y \neq \mu_y'\}$ where $\mu_y$ and $\mu_y'$ are two disintegrations).
	\label{thm:disintegrationunique}
\end{theorem}

\begin{theorem}
	$f:X \to \mathbb R$ is uniformly continuous $\iff$
	$\exists \ \omega : \mathbb R_+ \to \mathbb R_+$ , such that
	$\omega$ is increasing and $\lim_{x \to 0} w(x) = 0$ with
	$|f(x) - f(y)| \leq \omega(d(x,y)), \ \forall x,y \in X$.
	We call $\omega$ the modulus of continuity.
	\label{thm:mod_continuity}
\end{theorem}

\begin{definition}(Equicontinuous)
	For a metric space $X$, the sequence of functions
	$f_n:X\to \mathbb R$ is equicontinuous if
	$\forall \epsilon >0,\ \exists \delta >0: \ d(x,y) < \delta
		\implies d(f_n(x),f_n(y))<\epsilon$ for every $n \in \mathbb N$.
	\label{def:equicontinuous}
\end{definition}

\begin{definition}(Equibounded) We say that a sequence (or family)
	of functions $(f_n)$ is equibounded,
	if $\exists M > 0 \ : \ |f_n(x)|< M < +\infty \ \forall n \in
		\mathbb N$. In words, there is a value $M$ that bounds all functions
	in the sequence.
	\label{def:equibounded}
\end{definition}

\begin{theorem}(Arzelà-Ascoli)
	If $X$ is a compact metric space with $f_n$ equicontinuous and
	equibounded, then $\exists f_{n_k}\to_{\text{unif.}}f$, where
	$f$ is continuous.
	\label{thm:arzela-ascoli}
\end{theorem}

\begin{theorem}
	Let $(X,d)$ be metric space. Thus, if $X$ is compact, then $\mathrm{Lip}(X)$ is dense in $C(X)$.
	\label{thm:lipdense}
\end{theorem}
\begin{prf} (Proof from \citet{stackoverflow1})
	Let $g:X\to \mathbb R$ be a continuous function, then
	since $X$ is compact, $g$ is uniformly continuous.
	Therefore, for any $\varepsilon>0$, one can take a
	$\delta >0$ such that $d(x,y) < \delta$ implies
	$|g(x)-g(y)|<\varepsilon$. Now, let $M = \sup_{x}|g(x)|$
	and define
	\begin{equation*}
		f(x) :=
		\sup_y g(y) - \frac{2Md(x,y)}{\delta}
	\end{equation*}
	Now, note that $f$ is Lipschitz, since
	\begin{align*}
		f(x_1) - f(x_2) &= 
		\sup_y \left(g(y) - \frac{2Md(x_1,y)}{\delta}\right) -
		\sup_y \left(g(y) - \frac{2Md(x_2,y)}{\delta} \right)\\
		&\leq
		\sup_y \frac{2M(d(x_1,y)-d(x_2,y))}{\delta})
	\end{align*}
	By the triangle inequality, $d(x_1,y) - d(x_2,y) \leq d(x_1,x_2)$, then
	\begin{equation*}
		\sup_y \frac{2M(d(x_1,y)-d(x_2,y))}{\delta} \leq 
		\sup_y \frac{2Md(x_1,x_2)}{\delta} = 
		\frac{2Md(x_1,x_2)}{\delta}
	\end{equation*}
	The same argument is valid by exchanging $x_1$ and $x_2$, so $f$ has Lipschitz constant
	$\frac{2M}{\delta}$. Next, let's prove that $\sup_x |g(x) - f(x)| < \varepsilon$.

	A first point to notice is that $f(x)\geq g(x)$, since for $y=x$, we have $f(x) = g(x)$.
	For $d(x,y) \geq \delta$,
	\begin{equation*}
		f(x) = \sup_y g(y) - \frac{2M d(x,y)}{\delta}\leq \sup_y - 2M \leq -M \leq g(x)
	\end{equation*}
	Hence $f(x)\geq g(x) \geq f(x)$, so we obtain an equality.

	For $d(x,y) < \delta$,
	\begin{equation*}
		f(x) - g(x) = \sup_y g(y) - g(x) - \frac{2M d(x,y)}{\delta}
		\leq \varepsilon -\frac{2M d(x,y)}{\delta} < \varepsilon
	\end{equation*}
	We conclude that $0< f(x) - g(x) < \varepsilon$, so $\sup_x|f(x)-g(x)|< \varepsilon$.

\end{prf}


\section{Auxiliary - Inequalities}

\begin{lemma}(Inf-Sup Inequality)
	\begin{equation}
		|\inf_{x \in A} f(x) - \inf_{x \in A} g(x)| \leq
		\sup_{x \in A}|f(x)- g(x)|
	\end{equation}
	\label{lem:infsup_ineq}
\end{lemma}
\begin{prf}
	Let's write $\sup_{x \in A}f(x)$ as $\sup_A f$ for simplicity.
	Note that $f = f - g + g$, hence,
	\begin{align*}
		\sup_A f = \sup_A f - g + g & \leq
		\sup_A (f-g) + \sup_A g \implies   \\
		\sup_A f - \sup_A g         & \leq
		\sup_A f-g \leq \sup_A |f-g|       \\
	\end{align*}
	Using the same argument for $g$, we obtain that
	\begin{equation}
		|\sup_A f - \sup_A g| \leq \sup_A |f-g|
	\end{equation}

	Finally, note that
	\begin{align*}
		|\sup_A f - \sup_A g| =
		|\inf_A (-f) - \inf_A (-g)| & =
		|-\inf_A f + \inf_A g| =                                                \\
		                            & =|\inf_A f - \inf_A g | \leq \sup_A |f-g|
	\end{align*}
\end{prf}

\begin{lemma} (Minkowski's Inequality)
	Let $X$ be a measurable space, for $p \in [1,+\infty)$ and $f,g \in L^p(X)$. Therefore,
	\begin{equation}
		||f + g||_{L^p(X)} \leq
		||f||_{L^p(X)} + 
		||g||_{L^p(X)}
	\end{equation}
	Where $||f||_{L^p(X)}^p = \int_X |f|^p d\mu$.
	\label{lem:minkowski}
\end{lemma}

\begin{lemma}[Lemma de Gronwall]
	Seja $f:I\to \mathbb R$ uma função contínua e não negativa
	em um intervalo $I\subset \mathbb R$, tal que
	\begin{equation}
		f(t) \leq a +\left | \int_u^t b f(s) ds\right|
	\end{equation}
	para todo $t\in I$, com $a,b\geq 0$, e $u \in I$ constantes. Então
	\begin{equation}
		f(t) \leq a e^{b|t-u|}, \forall t \in I.
	\end{equation}
\end{lemma}