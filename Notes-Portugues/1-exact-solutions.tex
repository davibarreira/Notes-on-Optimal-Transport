\newpage
\section{Optimal Transport Problems with Exact Solution}

In many cases, obtaining the exact solution to an OT problem might not be possible,
thus requiring the use of methods to approximate the real solution. Yet,
there are situations where it's possible to obtain the exact OT plan. This section
explores some of these situations.

\subsection{1D Optimal Transport}

\label{sec:1dOT}
For $\mu,\nu \in P_p(\mathbb R)$, the Wasserstein has a closed form solution, which
relies on the pseudoinverse of the cumulative distribution function.
\begin{definition}
  Let $\mu \in \mathcal P(\mathbb R)$. The cumulative distribution function (CDF) is
  \begin{equation}
    F_\mu(x) := \mu((-\infty,x])
  \end{equation}
  Note that $F_\mu$ is a nondecreasing and right-continuous function.
  \label{def:cumulativefunction}
\end{definition}

\begin{definition}
  Given a nondecreasing and right-continuous function $F:\mathbb R \to [0,1]$,
  its pseudoinverse is
  \begin{equation}
    F^{-1}(x):= \inf\{
    y \in \mathbb R \ : \ F(y) \geq x
    \}
  \end{equation}
  \label{def:pseudoinverse}
\end{definition}
After introducing these definitions, we can present the formula for computing the Wasserstein
distance (Remark 2.30 on \citet{peyre2019computational}):
\begin{equation}
  W_p(\mu,\nu)^p = \int_0^1| F_\mu^{-1}(x)- F_\nu^{-1}(x) |^p dx
\end{equation}
Note that for $p=1$ and $\mu,\nu$ both atomless, then
there exists an optimal Monge map $T = F_\nu^{-1} \circ F_\mu$.

For the discrete 1-D distributions, an even simpler algorithm can be devised. Let
$\mu = \sum^n_i=1 u_i \delta_{x_i}$ and
$\nu = \sum^m_j=1 v_i \delta_{v_j}$, where
$x_1\leq x_2 \leq ... \leq x_n$ and
$y_1\leq y_2 \leq ... \leq y_m$. Consider that each position $x_i$ has mass $u_i$
and each position $y_j$ has capacity $v_j$. The optimal transport plan
consists of moving particle $x_i$
to the closest position $y_j$, until capacity $v_j$ is filled.

\begin{figure}[H]
  \centering
  \def\svgscale{0.6}
  \includesvg[inkscapelatex=false]{Figures/ot-1d-discrete.svg}
  \caption{Illustration of the algorithm for optimally transporting distribution $\mu$ in blue
  to distribution $\nu$ in red.}
  \label{fig:ot-1d-discrete}
\end{figure}

\subsection{Transport Between Discrete Measures}

Let $\mu$ be a finite discrete probability measure, hence
\begin{equation}
  \mu := \sum^n_{i=1} u_i \delta_{x_i}
\end{equation}
Where $\mathbf x = (x_1,...,x_n) \in \mathbb R^{n\times d}$
represent the location of each mass particle $i \in \{1,...,n\}$. Vector
$\mathbf u \in \mathbb R^{n\times 1}$, with components $u_i \in (0,1]$,
is the vector of weights, representing the amount of ``mass'' of each particle. Hence,
discrete measures might be represented by a vector $\mathbf x$ of positions, and
$\mathbf u$ of weights.

Now, suppose that both $\mu$ and $\nu$ are discrete measures. Let $\mathbf u \in \mathbb R^{n\times 1}$
and $\mathbf v \in \mathbb R^{m \times 1}$ represent the vector of weights, and
$\mathbf x \in \mathbb R^{n\times d}, \mathbf y\in \mathbb R^{m\times d}$ represent the positions of each particle
of $\mu$ and $\nu$, respectively.
In this scenario, the Optimal Transport Problem might be reformulated as the following.
The cost function $c(x,y)$ can be substituted by a cost matrix $\mathbf C \in \mathbb R^{n \times m}$, where
\begin{equation}
  \mathbf C_{i,j} := c(x_i,y_j), \quad i \in \{1,...,n\}, \ j \in \{1,...,m\}
\end{equation}
Any transport plan $\gamma$ can be written as a matrix $\mathbf P \in \mathbb R_+^{n\times m}$, such that
$\mathbf P_{i,j}$ represents the amount of mass flowing from particle $i$ to particle $j$. Since
$\gamma \in \Pi(\mu,\nu)$, the set of possible transport plans can be written as:
\begin{equation}
  \mathbf U(\mathbf u, \mathbf v)
  := \left\{
  \mathbf P \in \mathbb R_+^{n\times m} \ : \ \mathbf P \mathbf 1_m = \mathbf u , \
  \mathbf P^\mathrm T \mathbf 1_n = \mathbf v
  \right\}
\end{equation}

Where $\bm 1_n$ is a vector with $n$ components, each equal to 1. In words, the sum
of each row of $\mathbf P$ must be equal to $\mathbf u$ and the sum of each column must
be equal to $\mathbf v$.

The Kantorovich Problem can be written as:
\begin{flalign}
  \text{(KP-Disc.)}&&
  \mathrm{L}_{\mathbf C}(\mathbf{u,v}) :=
  \min_{\mathbf P \in
    \mathbf U(\mathbf u, \mathbf v)
  } \langle \mathbf C, \mathbf P \rangle =
  \min_{\mathbf P \in
    \mathbf U(\mathbf u, \mathbf v)}
  \sum_{i=1}^n \sum_{j=1}^m \mathbf C_{i,j} \mathbf P_{i,j} &&
  \label{eq:kpdisc}
\end{flalign}

The Dual Problem becomes:
\begin{flalign}
  \text{(DP-Disc.)}&&
  \mathrm{L}_{\mathbf C}(\mathbf{u,v}) :=
  \max_{\mathbf (\mathbf f,\mathbf g) \in
    \mathbf R(\mathbf C)
  }
  \langle \mathbf f, \mathbf u \rangle
  +
  \langle \mathbf g, \mathbf v \rangle
  &&
\end{flalign}

Where
\begin{equation}
  \mathbf R(\mathbf C) :=
  \left\{
  (\mathbf f, \mathbf g) \in \mathbb R^n \times \mathbf R^m \ : \
  \forall (i,j) \in \{1,...,n\} \times \{1,...,m\}, \
  \mathbf f \oplus \mathbf g \leq \mathbf C
  \right\}
\end{equation}

The Discrete Optimal Transport Problem is actually a Linear Programming (LP) problem. Hence,
one can rearrange Equation \eqref{eq:kpdisc} to the standard form of LP.
\begin{definition}
  (Optimal Transport as standard LP)
  \begin{mini*}
    {}{\mathbf c^\mathrm T \mathbf p}{}{}
    \addConstraint{
      \mathbf{Ap} =
      \begin{bmatrix}
        \mathbf u \\
        \mathbf v
      \end{bmatrix}
    } {}{}{}
    \addConstraint{}{\mathbf p \geq 0}{}{}
  \end{mini*}
  Where
  \begin{align*}
    \mathbf p :=
    \begin{bmatrix}
      \mathbf P_{1,1} \\
      \mathbf P_{2,1} \\
      \vdots          \\
      \mathbf P_{n,1} \\
      \mathbf P_{2,1} \\
      \vdots          \\
      \mathbf P_{n,m}
    \end{bmatrix}
    , \quad
    \mathbf c :=
    \begin{bmatrix}
      \mathbf C_{1,1} \\
      \mathbf C_{2,1} \\
      \vdots          \\
      \mathbf C_{n,1} \\
      \mathbf C_{2,1} \\
      \vdots          \\
      \mathbf C_{n,m}
    \end{bmatrix}
    , \quad
    \mathbf A := \begin{bmatrix}
      \mathbf 1_n^\mathrm T \otimes \mathbf I_m \\
      \mathbf I_n               \otimes \mathbf 1_m^\mathrm T
    \end{bmatrix}, \\
  \end{align*}
  Note that $\mathbf I_n$ stands for the identity matrix, and $\otimes$ is the tensor product.
  \label{def:lpformat}
\end{definition}

\begin{definition}
  (Optimal Transport Dual Problem as LP)
  \begin{mini*}
    {}{
      \begin{bmatrix}
        \mathbf u \\
        \mathbf v
      \end{bmatrix}^\mathrm T \mathbf h}{}{}
    \addConstraint{\mathbf A^\mathrm T \mathbf h \leq \mathbf c} {}{}{}
  \end{mini*}
  Where $\mathbf h = [f_1,\ldots,f_n,g_1,\ldots,g_m]^\mathrm T$, with $\mathbf c$ and $\mathbf A$ the same as in the primal LP.
  \label{def:lpdual}
\end{definition}

Since the Optimal Transport Problem is actually a Linear Programming Problem, therefore, one can use known methods for solving such
problems, such as Simplex or Interior Point Method. An explanation on how to solve such LP problems in OT can be found in Chapter
3 of \citet{peyre2019computational}.