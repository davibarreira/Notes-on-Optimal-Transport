\documentclass[10pt]{beamer}

% \usepackage[table,xcdraw]{xcolor}
\usepackage[square,numbers]{natbib}
\usepackage{graphicx}
\usepackage{svg}
\usepackage{pgfplots}
\usepgfplotslibrary{dateplot}

% \usetheme[progressbar=frametitle]{metropolis}
\usecolortheme{Imperial}
\usepackage{appendixnumberbeamer}

\usepackage{booktabs}
\usepackage[scale=2]{ccicons}


\usepackage{pgfplots}
\usepgfplotslibrary{dateplot}
\usepackage{xspace}
\newcommand{\themename}{\textbf{\textsc{metropolis}}\xspace}

\usepackage{amsfonts, amsmath, amsthm, amssymb}
\usepackage{bbm}
\usepackage{bm}
\usepackage{mathtools}
\usepackage[ruled,vlined]{algorithm2e}
\newlength{\commentWidth}
\setlength{\commentWidth}{7cm}
\newcommand{\atcp}[1]{\tcp*[r]{\makebox[\commentWidth]{#1\hfill}}}
\usepackage{setspace}
\usepackage[utf8]{inputenc}

\newcommand*{\QEDA}{\hfill\ensuremath{\blacksquare}}%
\newcommand*{\QEDB}{\hfill\ensuremath{\square}}%

\usepackage{tikz}
\usepackage{tikz-qtree}
\usepackage{forest}
\usetikzlibrary{trees} % this is to allow the fork right path

\setbeamertemplate{itemize items}[circle]
\setbeamertemplate{enumerate items}[default]
\usefonttheme[onlymath]{serif}


\title{\textbf{EDPs via Fluxo de Gradiente em \\Espaços de Wasserstein}}
\subtitle{}
% \date{\today}
\date{}
\author{\textbf{Autor:} Davi Sales Barreira \\\hfill\\}
\institute{\includegraphics[height=0.55cm]{emaplogo.png}}
\titlegraphic{\includegraphics[height=0.35cm]{emaplogo-neg.png}}


\usepackage{caption}
\captionsetup[figure]{font=footnotesize}

\begin{document}

\maketitle

\begin{frame}{Sumário}
  \setbeamertemplate{section in toc}[sections numbered]
	\setbeamertemplate{subsection in toc}[subsections numbered]
	% \setbeamerfont{section in toc}{size=\normal}
	\setbeamerfont{subsection in toc}{size=\small}
  % \tableofcontents[hideallsubsections]
  \tableofcontents[sectionstyle=show, subsectionstyle=show]
\end{frame}

\AtBeginSection{}
\section[Ideia Geral]{Ideia Geral e Motivação}
\begin{frame}[fragile]{Ideia Geral}

O espaço de Wasserstein se trata de um espaço métrico de medidas
de probabilidade embutido com a métrica de Wasserstein.
\vspace{3mm}

Um Fluxo de Gradiente é um sistema de equações onde a evolução do sistema se
dá através da descida de gradiente.
\vspace{3mm}

A ideia geral dessa apresentação é mostrar como algumas EDPs
podem ser reformuladas em termos de um Fluxo de Gradiente
em um espaço de Wasserstein. Apresentaremos como reformular a equação de calor, porém,
esse método é mais geral, sendo aplicável para muitas outras EDPs.

\end{frame}

\begin{frame}[fragile]{Motivação}

Por que interpretar EDPs como Fluxo de Gradiente em Wasserstein?
\vspace{3mm}
\begin{enumerate}
	\item Estética. Veremos que é uma bela interpretação que permite
	entender as EDPs de outro ponto de vista;
	\item Reformulação permite utilizar outros ferramentais
	para demonstrar, por exemplo, taxas de convergência,
	existência e unicidade;
	\item Esquema de discretização de fluxos de gradiente
	como algoritmo para aproximar soluções fracas
	para as EDPs.
\end{enumerate}

\end{frame}

\AtBeginSection{}
\section[Teoria de Transporte Ótimo]{Teoria de Transporte Ótimo}
\subsection[Teoria OT]{Monge \& Kantorovich}
\begin{frame}[fragile]{Teoria OT - Monge \& Kantorovich}

\textbf{Problema de Monge} -
Qual a maneira ótima de transporta massa de uma configuração
para outra?
\vspace{3mm}

\begin{figure}[H]
  \centering
  \def\svgscale{0.4}
  \includesvg[inkscapelatex=false]{Figures/mongeproblem.svg}
  \caption{Massa não pode ser separada.}
  \label{fig:mongeproblem}
\end{figure}

\textbf{Kantorovich Problem} -
Relaxação do problema original de Monge.
\vspace{3mm}

\begin{figure}[H]
  \centering
  \def\svgscale{0.4}
  \includesvg[inkscapelatex=false]{Figures/kantorovichproblem.svg}
  \caption{Massa pode ser separada.}
  \label{fig:kantorovichproblem}
\end{figure}

\end{frame}


\begin{frame}[fragile]{Teoria OT - Monge \& Kantorovich}
	
\begin{definition}[Problema de Monge]
	Dadas duas medidas de probabilidade $\mu \in \mathcal P(X)$,
  $\nu \in \mathcal{P}(Y)$ e uma função de custo
  $c:X\times Y \to[0,+\infty]$, resolva:
  \begin{flalign}
    (MP) &&
    \inf
    \left\{
    \int_{X} c(x,T(x))d\mu \quad : \quad
    T_\# \mu = \nu
    \right\}&&
  \end{flalign}

\end{definition}

\begin{figure}[H]
  \centering
  \def\svgscale{0.45}
  \includesvg[inkscapelatex=false]{Figures/monge_map_example.svg}
  \caption{Exemplo de dois problemas de Transporte Ótimo.}
  \label{fig:monge_map_example}
\end{figure}
	 
\end{frame}

\begin{frame}[fragile]{Teoria OT - Monge \& Kantorovich}

\begin{definition}[Acoplamento]
	Sejam $(X,\mu)$ e $(Y,\nu)$ espaços de probabilidade. Para
  $\gamma \in \mathcal{P}(X\times Y)$, dizemos que $\gamma$
  é um acoplamento de $(\mu,\nu)$ se $(\pi_X)_\# \gamma = \mu$
  e $(\pi_Y)_\# \gamma = \nu$. Chamamos $\Pi(\mu,\nu)$
  do conjunto de \textbf{Planos de Transporte}:
  \begin{equation}
    \Pi(\mu,\nu) :=
    \left \{
    \gamma \in \mathcal{P}(X \times Y) \ :
    \ (\pi_X)_\# \gamma = \mu \quad
    \text{and} \quad
    (\pi_Y)_\# \gamma = \nu
    \right \}
  \end{equation}
\end{definition}

	
\begin{definition}[Problema de Kantorovich]
  Dadas duas medidas de probabilidade $\mu \in \mathcal P(X)$,
  $\nu \in \mathcal{P}(Y)$ e a função de custo
  $c:X\times Y \to[0,+\infty]$, resolva:
  \begin{flalign}
    (KP) &&
    \inf
    \left\{
    \int_{X \times Y} c(x,y)d\gamma \ : \
    \gamma \in \Pi(\mu,\nu)
    \right\}&&
    \label{eq:KP2}
  \end{flalign}
  \label{def:KP}
\end{definition}

\end{frame}

\begin{frame}[fragile]{Teoria OT - Problema Dual}

	O Problema de Kantorovich tem uma formulação dual, que
	para certas condições de regularidade possui a mesma
	solução ótima que o problema primal (dualidade forte).

\begin{definition}[Problema Dual]
  Dadas $\mu \in \mathcal P(X)$, $\nu \in \mathcal P (Y)$ e
  custo $c:X \times Y \to \mathbb R_+$. O
  Problema Dual é
\begin{flalign}
  \mathrm{(DP)} &&
  \sup \left \{
  \int_X \phi \ d\mu + \int_Y \psi \ d\nu \ :
  \phi \in C_b(X) \ , \psi \in C_b(Y) \ ,
  \ \phi \oplus \psi \leq c
  \right \}
  &&
  \label{eqt:dualproblem}
\end{flalign}
\end{definition}

Funções $\phi, \psi$ são chamdas de \textbf{Potenciais de Kantorovich}.

\end{frame}

\begin{frame}[fragile]{Teoria OT - Problema Dual}

	When the cost function is a distance metric, the Dual Problem can be
	written in what is known as the Kantorovich-Rubinstein formulation.

\begin{theorem}[Kantorovich-Rubinstein]

  Let $(X,d)$ be a Polish space with metric $d$, and cost function $c(x,y) = d(x,y)$.
  Then, for $\mu, \nu \in \mathcal P(X)$, the Kantorovich Problem
  is equivalent to
  \begin{equation}
      \sup \left \{
      \int_X \phi \ d\mu - \int_X \phi \ d\nu \ :
      \phi \in Lip_1(X)
      \right \}
  \end{equation}
  \label{thm:Kantorovich-Rubinstein}
\end{theorem}

\end{frame}

\subsection{Wasserstein Distance}
\begin{frame}[fragile]{Optimal Transport Theory - Wasserstein Distance}

\begin{definition}[Wasserstein Distance]

  Let $(X,d)$ be a Polish metric space, with $c:X \times X \to \mathbb R$ such that $c(x,y)=d(x,y)^p$, and
  $p \in [1,+\infty)$.
  For $\mu,\nu \in \mathcal P_p(X)$, the Wasserstein Distance is given by:
  \begin{equation}
    W_p(\mu,\nu) :=
    \left(
    \inf_{\gamma \in \Pi(\mu,\nu)}
    \int_{X \times X} d(x,y)^p \ d\gamma
    \right)^{1/p}
    \label{def:Wasserstein}
  \end{equation}
\end{definition}

$\mathcal P_p(X)$ is the space of probability measures with finite $p$th moment.

\end{frame}

\begin{frame}[fragile]{Optimal Transport Theory - Wasserstein Distance}

The Wasserstein distance has many interesting properties which make it useful
in Machine Learning applications. Two of them that are of utmost interest
are the fact that it metrizes weak convergence and the
incorporation of the ground geometry.

\begin{figure}[H]
  \centering
  \def\svgscale{0.50}
  \includesvg[inkscapelatex=false]{Figures/wassersteingeometry.svg}
	\caption{Comparison between Wasserstein distance and KL Divergence, based on \citet{montavon2016boltzmann}.}
	\label{fig:wl-kl}
\end{figure}

\end{frame}

\begin{frame}[allowframebreaks]{References}
	\nocite{*}

% \renewcommand{\bibsection}{\section{}}
  \renewcommand{\section}[2]{}%
\tiny{\bibliography{ref}}
\bibliographystyle{plainnat}
  % \bibliographystyle{plain}
  % \bibliographystyle{abbrv}
  % \bibliographystyle{apa}
\end{frame}


\end{document}